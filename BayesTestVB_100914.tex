\documentclass[12pt,epsf]{article}
\usepackage{indentfirst}
\usepackage{array}

\usepackage[usenames]{color}
\usepackage{psfrag,graphicx}
\usepackage{eso-pic,graphicx}
\usepackage{colortbl}

%Sachins packages
\newcommand{\cN}{{\mathcal{N}}}
\newcommand{\cAL}{{\mathcal{AL}}}
\newcommand{\cH}{{\mathcal{H}}}
\newcommand{\cGa}{{\mathbf{Ga}}}

\newcommand{\cR}{{\mathcal{R}^\mathcal{D}}}
\newcommand{\cf}{{\mathbf{f}}}
\newcommand{\cfs}{\mathbf{f_\star}}
\newcommand{\cx}{\mathbf{x}}
\newcommand{\cX}{\mathbf{X}}
\newcommand{\cxs}{\mathbf{x}_\star}
\newcommand{\ct}{{\mathbf{t}}}
\newcommand{\ceps}{{\mathbf{\epsilon}}}
\newcommand{\cI}{\mathbf{I}}
\newcommand{\cA}{\mathbf{A}}
%\newcommand{\cx}{{\mathbf{x}}}
\newcommand{\cB}{\mathbf{B}}
\newcommand{\cD}{\mathbf{D}}
\newcommand{\czero}{{\mathbf{0}}}
\newcommand{\cK}{{\mathbf{K}}}
\newcommand{\cSigma}{{\mathbf{\Sigma}}}
\newcommand{\cem}{{\mathbf{\m}}}
\newcommand{\cmu}{{\boldsymbol{\mu}}}
\newcommand{\cexp}{{\text{exp}}}
\newcommand{\csI}{\sigma^2\cI}
\newcommand{\cts}{\cf}
\newcommand{\cData}{\mathcal{D}}
\newcommand{\cPhi}{{\mathbf{\Phi}}}
\newcommand{\Pzun}{p(z)}
\newcommand{\cpt}{\int_0^{\infty}\Pzun p(\ct|z)\,dz}
\newcommand{\cptb}{\int_0^{\infty}\Pzun~\cN(\ct| \czero,z\cK_{..})\,dz}
\newcommand{\cBess}{\mathcal{K}}
\newcommand{\cBessI}{\mathcal{I}}
\newcommand{\cint}{{\int_0^\infty}}
\newcommand{\cfKinvf}{\cf^T\cK^{-1}\cf}
\newcommand{\cGIG}{\mathbf{GIG}}

\newcommand{\GIG}{{\mathbf{GIG}}}
\newcommand{\cEz}{\left\langle\frac{1}{z}\right\rangle_{q(z)}}
\newcommand{\cEzz}{\left\langle\frac{1}{z}\right\rangle}
\newcommand{\cEfKinvf}{\left\langle\cfKinvf\right\rangle_{q(\cf)}}
\newcommand{\czKsI}{\frac{1}{\cEzz_{q(z)}}\cK+\csI}
%angles paranthesis
\newcommand{\clangle}{\left\langle}
\newcommand{\crangle}{\right\rangle}
\newcommand{\chalf}[1]{\frac{1}{2#1}}
\newcommand{\css}{\sigma^2}
\newcommand{\clrbracket}[1]{\left(#1\right)}
\newcommand{\clrangle}[1]{\clangle#1\crangle}

\newcommand{\ctKinvt}{\ct^T\cK_{..}^{-1}\ct}
\newcommand{\cp}{\tilde{p}}
\newcommand{\ca}{\tilde{\alpha}}
\newcommand{\cb}{\tilde{\beta}}
\newcommand{\cabt}{\ca\cb}

\newcommand{\cU}{\mathbf{U}}
\newcommand{\czKcsI}{z \cK+\csI}
\newcommand{\cKx}{\cK_{\cx,\cx}}
\newcommand{\cv}{\mathbf{v}}

\newcommand{\cS}{\mathbf{S}}
\newcommand{\cw}{\mathbf{w}}
\newcommand{\cz}{\mathbf{z}}
\newcommand{\cW}{\mathbf{W}}
\newcommand{\cy}{\mathbf{y}}
\newcommand{\calpha}{\bm{\alpha}}
\newcommand{\clambda}{\bm{\lambda}}
\newcommand{\cgamma}{\bm{\gamma}}
\newcommand{\ceta}{\bm{\eta}}
\newcommand{\cbeta}{\bm{\beta}}

\newcommand{\caexp}{\alpha(1-\alpha)}
\newcommand{\csum}{\sum_{i=1}^{N}}
\usepackage{amsmath}
\usepackage{amssymb}
\usepackage{bm}
\usepackage{natbib}
\usepackage{hyperref}
\usepackage{breqn}
\usepackage{refstyle}
\usepackage[toc,page]{appendix}
%\usepackage{graphicx}
%\DeclareGraphicsRule{*}{eps}{.bb}{}
\renewcommand{\baselinestretch}{1.5}
\setlength{\oddsidemargin}{0in} \setlength{\evensidemargin}{0in} \setlength{\textwidth}{6.5in}
\setlength{\textheight}{8.3in} \setlength{\headsep}{0in} \setlength{\parindent}{0.3in} \marginparwidth 0pt
\oddsidemargin 0pt \evensidemargin 0pt \marginparsep 0pt \topmargin 0pt \headheight 0pt \headsep 0pt \textheight
23.5 truecm \textwidth 16.0 truecm
\newtheorem{theorem}{Theorem}
\newcommand{\utwi}[1]{\mbox{\boldmath $ #1$}}
\newcommand{\ratio}{($\sigma_{\phi_i}/\tau_{i}, c_i$)}
\newcommand{\hs}{\hspace{1cm}}
\newcommand{\utheta}{\utwi{\theta}}
\newcommand{\ualp}{\utwi{\alpha}}
\newcommand{\calF}{{\cal F}}
\newcommand{\blue}[1]{\textcolor{blue}{$#1$}}
\newcommand{\red}[1]{\textcolor{red}{$#1$}}
\newcommand{\ignore}[1]{}{}
\newcommand{\cblue}{\textcolor{blue}}
\newcommand{\cred}{\textcolor{red}}
\definecolor{slight}{gray}{0.75}
\fboxsep=1pt
\def\r{\color{red}}
\def\cblue{\color{blue}}
\date{}


\begin{document}
%\title{\Large \bf Variational Bayes for Assessment of Dynamic Quantile Forecasts}
%
%\author{
%\normalsize  Richard Gerlach$^{1}$\thanks{Corresponding author: Richard Gerlach, Email: richard.gerlach@sydney.edu.au}
%and Sachin Abeywadena$^{2}$ \\
%\normalsize $^{1}$ The University of Sydney Business School, Australia. \\
%\normalsize $^{2}$ School of Engineering, The University of Sydney, Australia.}
%
%\maketitle

\noindent

\thispagestyle{empty}
\begin{abstract}
Methods for Bayesian testing and assessment of dynamic quantile forecasts have recently been proposed. Among these,
simple Bayes factor analogues of popular frequentist tests for accuracy of a time series of quantile forecasts have been developed.
To evaluate the relevant marginal likelihoods involved, either inappropriate assumptions of Gaussianity have been made, or
pre-packaged multivariate adaptive quadrature methods have been employed. This paper develops variational Bayes methods to
facilitate a more accurate version of one existing Bayesian test, and to develop a new test that out-performs its competing
Bayesian analogues. The size and power properties of the proposed methods are examined via a simulation study,
illustrating favourable comparisons with existing testing methods. An empirical study employs
the proposed methods, in comparison with standard tests, to assess the adequacy of a range of
forecasting models for Value at Risk (VaR) in several financial market data series.
\end{abstract} \noindent {\em Key words}:Bayesian Hypothesis
testing; Bayes factor; asymmetric-Laplace distribution; Value-at-Risk; quantile regression.

%\noindent {\em JEL Code:} Bayesian Hypothesis testing, MCMC, Value-at-Risk, quantile estimation.


%\newpage
%\setcounter{page}{2}

\section{Introduction}
%% Determining the accuracy of VaR models

Value-at-Risk (VaR) is now a standard measurement tool in risk management, used extensively by banking and other financial institutions.
It is an estimate (forecast) of the size of the minimum potential loss, over a given time horizon, with a specified probability,
for a financial position. VaR thus corresponds to the multiple of a quantile of the financial return distribution. It is a measure
that is widely used in practice for capital allocation, to protect against large negative market movements in asset prices,
following the Basel II Capital Accord. The accord further advises risk managers to regularly back-test their VaR forecast models,
using at least one year of historical data to compare actual returns with VaR forecasts.

Four well-known formal back-testing methods for quantile (VaR) forecasts are: the unconditional coverage (UC) test of Kupiec (1995); the
conditional coverage (CC) test of Christoffersen (1998); the dynamic quantile (DQ) test of Engle and Manganelli (2004); and the VaR Quantile
Regression (VQR) test of Gaglianone et al. (2011). Berkowitz, Christofferson and Pelletier (2011) developed a unified Lagrange
Multiplier framework for VaR assessment, incorporating these tests (except the VQR). Gaglianone et al. (2011) highlighted that the
VQR and DQ tests were over-sized in general, more-so in smaller samples, but have higher size-adjusted power than the UC and CC tests.
The latter outcome highlights that the VQR and DQ tests use more information, and in a mostly more effective manner, than the binary
variables that the UC and CC tests solely rely on; whilst the former result indicates a potential opportunity to develop better
and more accurate tests, which is the goal of this paper.

Bayesian methods for back-testing VaR forecasts are developed in Gerlach, Chen and Lin (2014), who develop a suite of
Bayesian methods that are roughly analogous to, and in simulations compared favourably with, the above four frequentist tests.
Their proposed methods are based on Bayes factors, which require estimation of marginal likelihoods; and such was achieved either
through making an inappropriate assumption of Gaussianity ("for simplicity") and using analytic integration; or by
employing adaptive quadrature methods, which are not standard in the estimation of marginal likelihoods and can be slow
in more than two dimensions and inaccurate, especially in highly concentrated likelihood situations.

The goal of this paper is to develop two further Bayesian testing methods for quantile forecast adequacy: one is a new test,
whilst the second attempts to improve on the accuracy of the quadrature-reliant test in Gerlach et al (2014). Both these tests
employ a method known as variational Bayes to estimate the required marginal likelihoods. Variational Bayes methods have become
popular in the literature for fast and accurate approximation of intractable likelihoods in Bayesian settings, e.g. see Ormerod and
Wand (2010). As with the UC, CC, DQ and VQR tests, and their Bayesian counter-parts, the proposed Bayesian tests will also not
depend on the model that generated the data, nor even on the method of estimation of the model parameters involved in forecasting.

The article is organized as follows: Section 2 reviews the evaluation of quantile forecast accuracy via existing frequentist
and Bayesian tests, as well as the variational Bayes method; Section 3 introduces the two new testing procedures;
Sections 4 and 5 present and discuss the results from a simulation study; Section 6 illustrates the results of an
empirical study comparing a range of competing VaR methods; concluding remarks appear in Section 7.

\section{Evaluating quantile forecast accuracy}
Let $y_t$ denote the return observation at time $t$, then VaR ($VaR_t$) at level $\alpha$ can be defined via:
\begin{eqnarray*}
\textrm{Pr}(y_t < -\mbox{VaR}_t|\calF_{t-1}) = \alpha,
\end{eqnarray*}
where $\calF_{t-1}$ is the information available at time $t-1$. For a forecast sample period, the
observed violation rate is the number of violations, i.e. return observations that are more extreme than their
respective VaR forecast (I($y_{t}< -\mbox{VaR}_{t})=1$), divided by the forecast sample size $m$.

\subsection{Frequentist dynamic quantile and quantile regression testing}
Engle and Manganelli (2004) develop the DQ test, a joint test for correct coverage and independence of violations, that employs more
than just the binary violation series. The null is $ H_{0} \,:\, I(y_{t}< -\mbox{VaR}_{t})$ are an i.i.d. series with rate $\alpha$.
A series of ``hits'', $H_t = I_{t}-\alpha$, are then calculated. Under the null it is
straightforward to show that $E(H_t) = 0$ and $E(H_tW_{it})=0$, where $W$ contains $q$ relevant explanatory variables in the information
set at time $t-1$. The DQ test statistic examines whether all parameters in a regression of $H$ on $W$ equal zero, calculated as:
\begin{eqnarray*}
\mbox{DQ}(q) = \frac{\utwi{H}'\utwi{W}\left(\utwi{W}'\utwi{W}\right)^{-1}\utwi{W}'\utwi{H}}{\alpha(1-\alpha)},
\end{eqnarray*}
which is analogous to a regression F statistic. Under the null, $DQ(q)$ tends to a $\chi^2_q$ distribution as $n$ gets large.
As in Engle and Manganelli (2004), we employ lagged hits and the VaR forecast itself as explanatory variable choices,
i.e. $W_t^{'} = (1, H_{t-1}, \mbox{VaR}_{t})$ (denoted as DQ1) and $W_t^{'} = (H_{t-1}, \ldots, H_{t-4}, \mbox{VaR}_{t}$ (denoted as DQ4).

Gaglianone et al. (2011) employed direct ``Mincer-Zarnowitz'' quantile regression for the $\alpha$th conditional quantile of $y_t$:
\begin{eqnarray}\label{qreg}
Q_{y_t } (\alpha |\calF_{t-1}  ) = \beta_0  + \beta_1 VaR_t, \quad{\textrm{for all }}\alpha  \in {\rm{(0,1)}},
\end{eqnarray}
their test employs the actual data, not violation indicators, as well as the VaR forecast series itself, thus employing more information
than the UC, CC and DQ tests. If the VaR forecasts are accurate, then $\beta _0  = 0, \beta _1  = -1$ in \ref{qreg}. Gaglianone et al. (2011)
test that hypothesis, i.e. $\theta= (\beta_0, \beta_1-1)^{'} = \bf{0}$, using the statistic
$\mbox{VQR} = \hat{\theta}^{'} \left(\hat{\Sigma} \right)^{-1} \hat{\theta}$,
which asymptotically follows a $\chi^2_2$ distribution, under the null hypothesis. We followed Gaglianone et al. (2011) and Koenker and
Machado (1999)'s recommendations here, in particular to estimate the matrix $\Sigma$. See those papers for details.

These are two of the most most powerful, best performing tests in the literature to assess the accuracy of quantile forecasts. In the next
section we review two of the most powerful Bayesian tests in Gerlach et al (2014).

\subsection{Bayes Factor Testing}
In a Bayesian framework, hypothesis testing and model comparison problems can be tackled via marginal likelihoods, that are often
translated into Bayes factors (BFs). A simple rule is that
$ p(y|M_k) = \int p(y|\theta, M_k) p(\theta | M_k) d \theta $, where model $M_k$ is generally preferred over $M_j$ if
$\mbox{BF} =\frac{p(y| M_k)}{p(y|M_j)} > 1$. BFs can also be employed in hypothesis testing of $\theta= \theta_0$,
where the hypothesis is rejected if $\frac{p(y|\theta_0, M)}{p(y|M)} < 1$.

\subsubsection{A BF Dynamic Quantile test}
A Bayes factor requires an assumed model and data distribution to produce a likelihood. The DQ test employs the series of ``hits'' $H_t =
I_{t}-\alpha$, $t=1,\ldots,n$ and fits a regression:
$$ H_t = \beta_0 + \sum_{i=1}^{(q-1)} \beta_i W_{i,t} + \epsilon_t$$
To get a likelihood, a distribution needs to assumed for $\epsilon_t$. The simplest, but non-intuitive, choice made by Gerlach et al (2014),
is $\epsilon_t \sim N(0,\sigma^2)$. This leads to:
$$p(H|\beta,\sigma^2) = (2 \pi)^{-0.5(n-q-1)} \sigma^{-\frac{n-q-1}{2}} \exp\left( -\frac{1}{2\sigma^2} \sum_{t=q+1}^n \epsilon_t^2\right) \,\, .$$
$\sigma^2$ is a nuisance parameter here, but under the standard Jeffreys' prior $p(\sigma^2) \propto \sigma^{-2}$ it can be
analytically integrated out, giving:
$$
\mbox{BFDQ} = \frac{ \left[ 0.5 \sum_{t=q+1}^n H_t^2\right]^{-m/2} }{\int \ldots \int \left[ 0.5 \sum_{t=q+1}^n \epsilon_t^2\right]^{-n/2}
p(\beta)  d \beta}.
$$
Under a proper Gaussian prior on $\beta$, e.g. $\beta|\sigma^2 \sim N(0,C \sigma^2)$ (where $C$ is a diagonal matrix with large elements),
the denominator can be integrated analytically (e.g. as in Smith and Kohn, 1996) and BFDQ calculated. Under the null all
$\beta = 0$, which is rejected whenever BFDQ$<1$. Gerlach et al (2014) employ the same regressors as for the DQ statistics,
giving BFDQ1 and BFDQ4 tests.

Naturally, many other possibilities and choices are possible and more sensible here, other than Gaussian errors. In the next
section we develop a new testing procedure that makes more reasonable assumptions, but has the same spirit as the DQ framework.

\subsubsection{A BF VQR test}
Koenker and Machado (1999) note that quantile regression estimation, usually performed by minimising the quantile distance function:
\begin{eqnarray*}
\stackrel{\rm min}{\utwi{\beta}} \,\, \sum_{t} u_{t} \left[\alpha-I(u_{t}<0)\right]
\end{eqnarray*}
is equivalent to a maximum likelihood (ML) estimation procedure when assuming i.i.d. skewed Laplace errors,
i.e. $u \sim SL(0,\sigma,\alpha)$, so that:
\begin{eqnarray*}
p_{\alpha}(u)=\frac{\alpha(1-\alpha)}{\sigma}\exp\left[-\left (\frac{u\left[\alpha-I(u<0)\right]}{\sigma}\right)\right].
\end{eqnarray*}
The ML and usual quantile regression estimates for $\utwi{\beta}$ are mathematically equivalent in this case.

The quantile regression model for the $\alpha$th conditional quantile of $y_t$, regressed against its VaR forecast, can be written:
\begin{eqnarray*}
Q_{y_t } (\alpha |\calF_{t-1}  ) = \beta_0  + \beta_1 VaR_t, \quad{\textrm{for all }}\alpha  \in {\rm{(0,1)}}.
\end{eqnarray*}
To develop a Bayesian VQ procedure, again assuming a Jeffreys prior on $\sigma$ and integrating it out, Gerlach et al (2014) found:
\begin{eqnarray*}
p(\utwi{u}|\utwi{\beta}) = \alpha^n(1-\alpha)^n \Gamma(n) \left[\sum^{n}_{t=1} u_{t}(\alpha-I(u_t<0)) \right]^{-n} \,\, .
\end{eqnarray*}
Thus, the BFVQ statistic they developed is:
$$
\mbox{BFVQ} = \frac{ p({\bf u}|\beta_0=0, \beta_1=1)}{ \int \int p({\bf u}|\utwi{\beta}) p(\utwi{\beta}) d \beta_0 d \beta_1 }
$$
where we reject the null of $\beta_0=0, \beta_1=1$ whenever $BFVQ<1$. The denominator above is a double integral over the
bivariate real line. Gerlach et al (2014) employ a highly diffuse, but proper, Gaussian prior on $\beta$, then transform to
the region $(-1,1)\times(-1,1)$ and use adaptive quadrature methods to numerically estimate this integral; they did not
assess the accuracy of this method.

\subsection{Variational Bayes}
\label{sec:VB}
Generally, the marginal likelihood $p(\cy)$ required in the BF is not analytically tractable. VB methods can be used to estimate a
lower bound for this quantity, optimised to be as close as possible to $p(\cy)$ under certain assumptions. In such cases, the
marginal log-likelihood, $\log p(\cy)$, can be expressed as:
$\mathcal{L}(q(\cz),\theta)+KL(q(\cz)||p(\cz|\cy))$ where, $\mathcal{L}(q(\cz),\theta)=\int\int q(\cz)\log\frac{p(\cz,\cy)}{q(\cz)}d\cz$ is the expected joint log-likelihood of a set of relevant latent variables $\{\cz\}$ and $\cy$, with respect to a proposal distribution $q(\cz)$ and $\theta$ are the hyper-parameters. 
Here, $KL$ is the Kullback-Leibler divergence between the proposal distribution, and the posterior distribution of the latent variables. Note that $\mathcal{L}(q(\cz),\theta)$ is a lower bound to the true log data likelihood since $KL>0$. The Expectation-Maximisation (EM) algorithm proceeds by initially minimizing the KL divergence for a given set of hyper parameters, $\theta$ (i.e. finding an appropriate $q(\cdot)$). Usually, this is done by
setting $p(\cz|\cy,\theta)=q(\cz)$, in which case $\log p(\cy|\theta)=\mathcal{L}(q(\cz),\theta)$.

However, an appropriate analytic distribution for the full posterior $p(\cz|\cy)$ often cannot be found. Variational Bayes (VB)
methods (see Tzikas, Likas and Galatsanos (2008) and
Ormerod and Wand (2010)) differs from EM in that it focuses far more in maximising the lower bound $\mathcal{L}(q(\cz),\theta)$ and not on minimising the KL divergence. Additionally, to gain tractability of $\mathcal{L}(q(\cz),\theta)$ it is often assumed that the proposal distribution is factorisable such that, $q(\cz_1,\cz_2;\theta)=q_1(\cz_1)q_2(\cz_2)\approx p(\cz_1,\cz_2|\cy,\theta)$.
It can be shown that for each proposal distribution $\log q_i(\cz_i) \propto E_{q(\cz_{/i})}\clrbracket{\log p(\cy | \cz_i,\cz_{/i},\theta)+\log p(\cz_i)}$ where $\cz_{/i}$ is all latent variables but the i-th and, the expectation $E$ is taken with respect to all $q(\cz_j)$ but the i-th. Due to the dependence of the i-th proposal distribution on the rest of the proposal distributions, this algorithm is iterated until convergence.


\section{Two new Bayesian quantile testing procedures}
This section presents two new procedures that employ VB to estimate the required marginal likelihoods.

\subsection{A new Bayesian DQ procedure}
Engle and Manganelli (2004) employ a standard OLS estimation algorithm to a binary series of "hits". In such a setting, it is more
typical to prefer a logistic or probit regression formulation. The standard probit specification allows a
convenient VB representation, as in Girolami and Rogers (2006), as such we prefer it. Keeping the DQ test notation
above, a probit specification is:
$$ Pr\clrbracket{I_t=1 | \calF_{t-1} = \Phi \left(\beta_0 + \sum_{i=1}^{(q-1)} \beta_i W_{i,t}\right)},$$
with associated likelihood function:
$$
p(\bf{I}|\utwi{\beta}) = \Pi_{t=q+1}^{n} \Phi\left(W_t^{'}\utwi{\beta} \right)^{I_t} \left(1-\Phi\left(W_t^{'}\utwi{\beta} \right)\right)^{1-I_t}.
$$

Under the null of independent violations with coverage $\alpha$, the intercept is
$\beta_0 = \log\left(\frac{\alpha}{1-\alpha}\right)$ and all slope parameters $\beta_i = 0$. Thus, the relevant BF here is:
\begin{eqnarray*}
\mbox{VBDQ} = \frac{ p(\bf{I}| \beta_0 = \log\left(\frac{\alpha}{1-\alpha}\right), \beta_i=0)}
              { \int \ldots \int p(\bf {I}|\utwi{\beta}) p(\utwi{\beta}) d \beta_0 d \beta_1 \ldots d \beta_{q-1} }
\end{eqnarray*}
where we reject the null of $\beta_0=\log\left(\frac{\alpha}{1-\alpha}\right), \beta_i=0$, $i=1,\ldots,q-1$
whenever $VBDQ<1$.

The denominator here is a multiple integral over the real line in $q$ dimensions. As in Gerlach et al (2014), we
employ a highly diffuse, but proper, Gaussian prior on $\beta$; however the integral is still intractable. Employing the
usual Gaussian latent variable representation of the probit specification, the VB lower bound
estimate of Girolami and Rogers (2006) is employed, as derived in \secref{class_beta} and \secref{class_y}, yielding
the lower bound as:
\begin{align}
-\chalf{}\clrbracket{N\log 2\pi+d\log1000+\frac{\cmu^T\cmu+Tr(\cSigma)}{1000}+
\sum_{t=1}^{N}\cW_t^T\cSigma\cW_t-d-\log|\cSigma|-\sum_{t=1}^{N}\log C_t} \,\, ,
\end{align}
where all terms are defined in \secref{class_beta} - \secref{class_lik}.

\subsection{VB approximation to BFVQ statistic}
\label{sec:vb_derivation}
The BFVQ statistic in Gerlach et al (2014) (and above) is:
$$
\mbox{BFVQ} = \frac{ p({\bf u}|\beta_0=0, \beta_1=1)}{ \int \int p({\bf u}|\beta) p(\beta) d \beta_0 d \beta_1 }
$$
The denominator here is intractable. We propose to approximate this integral by the method of VB.

We begin by restating the Asymmetric Laplace distribution as a mixture of a Gaussians. Thus the Bayesian hierarchy is as follows:
\begin{align}
\nonumber p(y_i|\cbeta,w_i,\sigma,\alpha)&=\cN\clrbracket{\cbeta^T\cW_i+\frac{1-2\alpha}{\alpha(1-\alpha)}\sigma w_i,
\frac{2}{\alpha(1-\alpha)}\sigma^2 w_i}\\
\nonumber p(\cbeta)&=\cN(0,1000\cI)\\
\nonumber p(w_i)&=\exp(-w_i)\\
p(\sigma)&\propto\frac{1}{\sigma}
\label{eq:BFVQ_hier}
\end{align}

For brevity of notation, subsequently the notation $\clrangle{\cdot}$ is employed to denote an expectation under $q(\cdot)$.
The approximate distribution $q(\cbeta)=\cN(\cmu,\cSigma)$ where \footnote{Derivation is shown in \secref{q_beta}},
\begin{align}
\cSigma &=\clrbracket{\frac{\caexp}{2}\clrangle{\frac{1}{\sigma^2}}\csum\clrangle{\frac{1}{w_i}}\cW_i\cW_i^T+\frac{1}{1000}\cI}^{-1}\\
\cmu&=\cSigma\clrbracket{\csum\clrbracket{\frac{\caexp}{2}\clrangle{\frac{1}{\sigma^2}}\clrangle{\frac{1}{w_i}}y_i-\frac{1-2\alpha}{2}\clrangle{\frac{1}{\sigma}}}\cW_i}
\end{align}

The approximate distribution $q(\cw)$ can be factorised as independent components such that $q(\cw)=\prod_{i=1}^{N}q(w_i)$, where
$q(w_i) = \cGIG\clrbracket{\chalf{},\alpha,\beta}$ where,
\begin{align}
\alpha_i & =\clrbracket{\frac{(1-2\alpha)^2}{2\caexp}+2} \\
\beta_i & = \frac{\caexp}{2}\clrangle{\frac{1}{\sigma^2}}\clrbracket{y_i^2-2y_i\cW_i^T\clrangle{\cbeta}+\cW_i^T\clrangle{\cbeta\cbeta^T}\cW_i}
\end{align}

The exact VB distribution over $q(\sigma)$, though tractable, is susceptible to numerical problems due to the dependence of the
normalising constant on the parabolic cylindrical function. Hence, we instead limit $q(\sigma)$ to a Inverse Gamma
distribution, $IG(a,b)$. In order to obtain the parameters $a,b$ we are required to maximise the following function
using the given derivatives \footnote{Full derivation is shown in \secref{q_sigma}}.

\begin{align}
F_\sigma&=(a-N)(\log b-\psi(a))+(b-\gamma)\frac{a}{b}-\delta \frac{a(a+1)}{b^2}-a\log b+\log \Gamma(a)\\
\frac{\partial F}{\partial a}&=(N-a)\psi^{(1)}(a)-\frac{\gamma}{b}-\frac{\delta(2a+1)}{b^2}+1\\
\frac{\partial F}{\partial b}&=-\frac{N}{b}+\frac{\gamma a}{b^2}+\frac{2\delta a(a+1)}{b^3}
\end{align}

As stated previously the expected log likelihood of the set of latent variables $\mathcal{L}$ is a lower bound for the
marginal likelihood. Thus, using the approximate distributions derived, it can be shown that
\begin{align}
\nonumber &-\chalf{}\Biggl[N\log\frac{2}{\caexp} +2N(\log b -\psi(a))+D\log (1000)-\cmu^T\cSigma^{-1}\cmu\\
\nonumber&+\frac{\caexp}{2}\frac{a(a+1)}{b^2}\sum_{i=1}^{N}\sqrt{\frac{\alpha_i}{\beta_i}}y_i^2+\clrbracket{\frac{(1-2\alpha)^2}{2\caexp}+2}\csum\clrbracket{\sqrt{\frac{\beta_i}{\alpha_i}}+\frac{1}{\alpha_i}}\\
&-(1-2\alpha)\frac{a}{b}\csum y_i\Biggr]
+\chalf{}\clrbracket{\log|\cSigma|}-\chalf{}\csum\log\alpha_i+\frac{N}{2}+\log \Gamma(a)-a\psi(a)+a
\end{align}
is a lower bound \footnote{Full derivation can be seen in \secref{l_bound}}.

%$q(\cbeta,\cw, \sigma)=q(\cbeta)q(\cw)q(\sigma)\approx p(\cbeta,\cw, \sigma|\cy)$ where $\cw$ is an extra set of latent variables
%which will be introduced in \secref{vb_derivation}. As the KL divergence is always positive, $\mathcal{L}(q(\cbeta,\cw, \sigma),\theta)$
%forms a lower bound. In the M-step this lower bound $\int\int q(\cbeta,\cw, \sigma|\theta^{OLD})\log p(\cbeta,\cw, \sigma,\cy)d\cbeta dz$
%is maximised w.r.t. the hyper-parameters. The E and M steps are iterated over until convergence. However, since all possible parameters
%are marginalised over here, there will be no M-step.
%
%There exists a closed form solution for $q(\cz_i)$ under VB where $\cz_i$ is the i-th member of the set of latent variables $\cz$.
%Thus we let $q(\cz_i)=\frac{1}{Z}\exp(E_{q(\cz_{\backslash i})}[p(\cz,\cy)])$ where $\cz_{\backslash i}$ is the set of all
%latent variables except for the i-th member and $Z$ is the normalising constant. Thus under VB, the approximate distribution
%is the expectation of the joint probability between all latent variables and the observed values, w.r.t. the other
%approximate distributions. Since, an approximate distribution on one latent variable depends on the other
%approximate distributions, this step is iterated until convergence.


\section{Simulation study}
The empirical properties of the proposed Bayesian methods are assessed via a simulation study. The simulation setting as in
Gaglianone et al. (2011) and Gerlach et al (2014) is employed here. The true model is a GARCH(1,1), specified as:
\begin{eqnarray*}
\sigma_t^2 = 0.1 + 0.1y_{t-1}^2 + 0.85\sigma_{t-1}^2 \,\,;\,\, y_t = \sigma_t \epsilon_t \,;\, \epsilon_t \sim N(0,1) \\
\end{eqnarray*}
where $\mbox{VaR}_{t,\alpha} = \sigma_t \Phi^{-1}(\alpha)$. To assess power an incorrect, but common, historical simulation (HS)
VaR estimator is employed:
\begin{eqnarray*}
\mbox{HS}250_{t,\alpha} = \hat{Q}_{\alpha}(y_{t-250},\ldots,y_{t-1})
\end{eqnarray*}
using the sample percentile of the last 250 observations as a 1-step-ahead VaR forecast. 10000 replications of data, using sample sizes
$n=250, 500, 1000$ and $2500$, are simulated from this model. For each data set, the DQ1, DQ4 and VQR tests are conducted. Further, the
BFDQ1, BFDQ4 and BFVQ statistics from Gerlach et al (2014) are all calculated. Finally, the proposed VBDQ and VBVQ statistics are
also calculated for each data set. These statistics are all calculated under the null, using the true $\mbox{VaR}_{t,\alpha}$ series,
and then also calculated under the alternative, using the estimated $\mbox{HS}250_{t,\alpha}$ series. $\alpha=0.05, 0.01$ are used for the
quantile levels.

As in Gerlach et al (2014), to compare the methods on an equal footing we consider size and power as well as empirically adjusted
size and size-adjusted power. This is standard practice when comparing frequentist tests, but is not standard for Bayesian
methods; however this will allow direct, objective comparison of all methods considered on an equitable basis.

Table \ref{size5} shows the empirical estimates for size, as well as adjusted size, across the methods employed at
$\alpha=0.05$ for $n=250,500$. Also shown are the empirical $5\%$ points ("threshold") for all methods, calculated via
sample percentiles across the 25000 replications for each test statistic. These are the thresholds used to calculate the
empirically adjusted size and size-adjusted power below; i.e. adjusted size is the observed percentage
of test statistics, across the 25000 replications, that are beyond the empirical $5\%$ threshold calculated under the null hypothesis.
Size-adjusted power is the same observed percentage, using the same empirical threshold, employing the test statistics calculated under the
incorrect HS VaR estimator. There is no reason why the value 1 should be the 95th percentage point for the sampling distribution of any
BF statistic, so size for the BF methods is not particularly relevant, but is reported as a reference for comparison.
Adjusted size is, however, relevant to the comparison of all methods.

\begin{table}[thp]
\begin{center}
\caption{Size, empirical threshold and adjusted size for nominal 5\% tests of 1-step-ahead quantile forecasts at $\alpha=0.05$.
} \label{size5}
\begin{tabular}{lcccccc}
\hline
 & \multicolumn{3}{c}{n=250} & \multicolumn{3}{c}{n=500}                    \\
Method   &      Size  &          threshold &   adj. size &       Size &         threshold &  adj. size\\ \hline
DQ1      &\fbox{0.053}&              7.944 &\fbox{0.0500}&\fbox{0.049}&             7.760 &\fbox{0.050}   \\ [1.3pt]
BFDQ1    &      0.003 &             0.00024&\fbox{0.0500}&      0.0002&6.12$\times10^{-5}$&\fbox{0.050}   \\ [1.3pt]
VBDQ     &      0.000 &             0.00023&\fbox{0.0500}&      0.000 &9.23$\times10^{-5}$&\fbox{0.050}   \\ [1.3pt]
DQ4      &      0.062 &             13.421 &\fbox{0.0500}&      0.056 &            12.968 &\fbox{0.050}   \\ [1.3pt]
BFDQ4    &      0.001 &1.314$\times10^{-7}$&\fbox{0.0500}&      0.0001&7.95$\times10^{-9}$&\fbox{0.050}  \\ [1.3pt]
VQR      &  \cblue{0.044}&              5.554 &\fbox{0.0500}&  \cblue{0.054}&             6.274 &\fbox{0.050}   \\ [1.3pt]
BFVQ     &      0.191 &            117.952 &\fbox{0.0500}&      0.162 &           110.931 &\fbox{0.050}   \\ [1.3pt]
VBVQ     &      0.006 &            0.00037 &\fbox{0.0500}&     0.0018 &1.08$\times10^{-6}$&\fbox{0.050}   \\ [1.3pt]
\hline
\end{tabular}
\par\smallskip
\parbox{.9\textwidth}{ The correct, nominal thresholds are: VQR 5.99; DQ1 7.81; DQ4 12.59}
\end{center}
\end{table}

At $n=250$, only the DQ1 and VQR tests achieve close to a true nominal size, with $5.3\%, 4.7\%$ respectively,
whilst the DQ4 is quite over-sized and most BF methods are well under-sized; however all methods
achieve correct adjusted sizes of exactly $5\%$. For $n=500$ the DQ1 test again has the closest to nominal
size ($4.9\%$), followed by VQR and DQ4, again all methods achieve the corrected size exactly equal to nominal.

Table \ref{size51} shows the empirical estimates for size and then adjusted size across all the methods employed at $\alpha=0.05$
for $n=1000,2500$. Also shown are the empirical $5\%$ points for the methods across the 25000
replications used to calculate the adjusted size and size-adjusted power.

\begin{table}[thp]
\begin{center}
\caption{Size, empirical threshold and adjusted size for nominal 5\% tests of 1-step-ahead quantile forecasts at $\alpha=0.05$.
} \label{size51}
\begin{tabular}{lcccccc}
\hline
 & \multicolumn{3}{c}{n=1000} & \multicolumn{3}{c}{n=2500}                    \\
Method   &       Size  &            threshold &   adj. size &       Size  &           threshold &    adj. size   \\ \hline
DQ1      &\cblue{0.0450}&               7.577 &\fbox{0.0500}&  \cblue{0.0494}&               7.782 &\fbox{0.0500}   \\ [1.3pt]
BFDQ1    &       0.000  & 1.647$\times10^{-5}$&\fbox{0.0500}&      0.000  &4.062$\times10^{-6}$ &\fbox{0.0500}   \\ [1.3pt]
VBDQ     &       0.000  & 4.124$\times10^{-5}$&\fbox{0.0500}&      0.000  &1.647$\times10^{-5}$ &\fbox{0.0500}   \\ [1.3pt]
DQ4      &\cblue{0.0524}&              12.752 &\fbox{0.0500}&\fbox{0.0496}&              12.570 &\fbox{0.0500}   \\ [1.3pt]
BFDQ4    &       0.000  &7.060$\times10^{-10}$&\fbox{0.0500}&      0.000  &3.569$\times10^{-11}$&\fbox{0.0500}   \\ [1.3pt]
VQR      & \fbox{0.0510}&               6.034 &\fbox{0.0500}&  \cblue{0.0485}&               5.910 &\fbox{0.0500}   \\ [1.3pt]
BFVQ     &       0.1182 &              47.731 &\fbox{0.0500}&  \cblue{0.0480}&               0.848 &\fbox{0.0500}   \\ [1.3pt]
VBVQ     &       0.0001 &6.067$\times10^{-12}$&\fbox{0.0500}&      0.000  &4.124$\times10^{-27}$&\fbox{0.0500}   \\ [1.3pt]
\hline
\end{tabular}
\par\smallskip
\parbox{.9\textwidth}{ The correct, nominal thresholds are UC, IND 3.84; CC, VQ 5.99; DQ1 7.81; DQ4 12.59}
\end{center}
\end{table}

At $n=1000$, the VQR is closest to nominal size with $5.1\%$; all methods achieve corrected sizes exactly equal to $5\%$.
For $n=2500$ the DQ4 and DQ1 tests have the closest to nominal sizes ($4.96, 4.94\%$ respectively) and for all methods the corrected size is exactly
the nominal $5\%$ level.

\begin{figure}[thp]
     \centering
      \includegraphics[width=1.0\textwidth]{DQvsBFDQ1vsVBDQ.png}
\caption{\label{powerDQBF250} Two times the logarithm of the VBDQ (dark grey diamond), BFDQ1 (light grey circle) and DQ1 (black square) statistics against number of violations under the HS quantile estimator at $n=250$; (a) 5\% HS; (b) 1\% HS. The horizontal lines are two times the logarithm of the empirical 5\% points of the VBDQ (grey), DQ1 (black) and BFDQ1 (grey) statistics.}
\end{figure}

Table \ref{power5} shows the empirical estimates for power and size-adjusted power across the methods
employed at $\alpha=0.05$. At $n=250$ Gerlach et al (2014) found that the BFVQ was marginally more powerful, size-adjusted, than
the DQ1 and DQ4 tests, we replicate this result. However, the VBDQ statistic outperforms all tests and has the highest size-adjusted
power here. To examine this result in more detail, consider Figure \ref{powerDQBF250}(a), showing two times the logarithm of
the VBVQ, BFDQ1 and DQ1 statistics, plot against the number of violations from the incorrect HS 5\% VaR estimator. As expected, all three
methods reject the HS estimator for very low and very high numbers of violations (i.e. we expect 12.5 violations). However, these tests also
have plenty of rejections for violation numbers close to 12 (e.g. 9,15): for violation series showing "significant" correlation.
Gerlach et al (2014) found that the extra power of BFDQ1 over DQ1 is achieved through rejecting many more series when the number of
violations is comparatively low, specifically for 3-15 violations; the DQ1 has more rejections than BFDQ1 when 16-27 violations are observed,
but the differences in rejection frequencies here are much smaller than they are for 3-15 violations; as shown in Figure \ref{pdqBF250}(a)
which shows the rates of rejection at each number of violations for the three methods. Here the superior size-adjusted power of the VBDQ
is well displayed: from 1-15 violations the VBDQ mimics the higher rejection rates of the BFDQ1 statistic; from 16-27 violations the VBDQ
mimics the higher rejection rates achieved by the DQ1 statistic; for violation numbers of 1,2 or those higher than 28 all three methods always
reject the null. Thus, at $n=250$ and $5\%$ VaR forecasting, the VBDQ and BFDQ1 tests appear to have
much higher power than the DQ1 at detecting correlation in the violation series when the number of violations is relatively small
($\sim 3-15$ violations), however the VBDQ and DQ1 statistics have similarly higher size-adjusted power than the BFDQ1 when the number
of violations is relatively large (16-27 violations); thus contributing to an overall much higher size adjusted power in Table \ref{power5}
when $n=250$ for the VBDQ method.

As an aside, the VBVQ method had much higher size-adjusted power than both the BFVQ and VQR tests at $n=250$.


\begin{table}%[thp]
\begin{center}
\caption{Power and size-adjusted power for nominal 5\% tests of 1-step-ahead quantile forecasts at $\alpha=0.05$.
}\label{power5}
\begin{tabular}{lcccccccccccc}
\hline
 & \multicolumn{2}{c}{n=250} & \multicolumn{2}{c}{n=500} & \multicolumn{2}{c}{n=1000} & \multicolumn{2}{c}{n=2500}                   \\
Method   &      Power  &   Size-adj. &       Power &   Size-adj. &       Power &  Size-adj.  &       Power & Size-adj.    \\ \hline
DQ1      &\cblue{0.354}&       0.345 &\cblue{0.451}&       0.454 &\cblue{0.628}&\cblue{0.645}&\cblue{0.945}&\cblue{0.946} \\ [1.3pt]
BFDQ1    &       0.036 &\cblue{0.427}&       0.020 &\cblue{0.497}&       0.027 &\cblue{0.680}&       0.105 &\cblue{0.954} \\ [1.3pt]
VBDQ     &       0.000 & \fbox{0.500}&       0.000 & \fbox{0.515}&       0.000 &\cblue{0.640}&       0.000 &       0.924  \\ [1.3pt]
DQ4      &\cblue{0.380}&       0.351 &\cblue{0.510}&\cblue{0.493}& \fbox{0.707}&\cblue{0.700}&\cblue{0.967}&\cblue{0.967} \\ [1.3pt]
BFDQ4    &       0.016 &       0.371 &       0.010 &\cblue{0.507}&       0.013 & \fbox{0.711}&       0.061 & \fbox{0.971} \\ [1.3pt]
VQR      &       0.158 &       0.168 &       0.269 &       0.258 &       0.483 &       0.480 &\cblue{0.875}&       0.879  \\ [1.3pt]
BFVQ     & \fbox{0.558}&       0.304 & \fbox{0.566}&       0.328 &\cblue{0.648}&       0.433 &\cblue{0.916}&       0.923  \\ [1.3pt]
VBVQ     &       0.175 &\cblue{0.398}&       0.102 &       0.416 &       0.059 &       0.582 &       0.019 &       0.919  \\ [1.3pt]
\hline
\end{tabular}
\end{center}
\end{table}

At $n=500$, the four methods with highest size-adjusted power for 5\% quantile forecasting are: VBDQ, BFDQ4, BFDQ1 and the DQ4 test
with $\sim 50-52\%$, followed by the DQ1 with $45\%$ and then VBVQ with $42\%$. The VBDQ is only marginally preferred here and the
top four tests are very close in size-adjusted power. The BFDQ4 (BFDQ1) test rejects more than the DQ4 (DQ1) test when the number
of violations is between 8 and 27 (12 and 28; 25 violations are expected), whilst the DQ4 rejects more for 28-46 (DQ1 for 29-42)
violations: again the BFDQ statistics have slightly more power to detect correlated violations at lower numbers of violations, compared to
the DQ tests; however, in this case things even out so that the BFDQ and DQ method's size-adjusted powers are close to comparable
in each case. The VBDQ method has rejection rates similar to those of the BFDQ1 for lower violations and closer to that of
the DQ1 for higher violations, allowing it to have marginally the highest size-adjusted power overall.

For $n=1000$, the BFDQ4 has marginally the highest size-adjusted power, followed closely by the DQ4 and BFDQ1, then
DQ1 and VBDQ methods. Figure \ref{pdq1000} shows rejection rates for the VBDQ, BFDQ1, BFDQ4, DQ1 and DQ4 methods.
The BFDQ4 (BFDQ1) test rejects more than the DQ4 (DQ1) test when the number of violations is between
33 and 52 (33 and 55; 50 violations are expected), whilst the DQ4 (DQ1) rejects more than the BFDQ4 (BFDQ1) for 53-73 (DQ1 for 56-71)
violations: again the BFDQ statistics have slightly more power to detect correlation at lower numbers of violations, compared to the
DQ tests, with the result reversed for higher violation numbers. The VBDQ method has rejection rates marginally lower than
those of the BFDQ1 for most violations (36 to 63) and closer to that of the DQ1 for a small number of higher violations (64-69),
meaning it overall has marginally lower size-adjusted power than the BFDQ and DQ tests here. Figure \ref{pvq1000} shows rejection
rates for the VBVQ, BFVQ and VQR methods, plot (a) is for $5\%$. Clearly, at almost all violation numbers the VBVQ has a
marginally higher rejection rate, whilst BFVQ is superior to VQR at lower violations, but inferior at higher violation numbers.

For $n=2500$, similar results to $n=1000$ are obtained, but with a marked increase in size-adjusted power for all the
methods. Again the BFDQ4 methods has marginally the highest size-adjusted power of 97.4\%,
closely matched by DQ4 and then followed closely by BFDQ1, DQ1 and BFVQ, which marginally out-performs the VQR.

\begin{figure}%[htp]
     \centering
      \includegraphics[width=0.8\textwidth]{pdqBFVB250.png}
\caption{\label{pdqBF250} Rates of rejection for the VBDQ (dark grey diamond), BFDQ1 (light grey circle) and DQ1 (black square) statistics against the
number of violations under the HS quantile estimator at $n=250$; (a) 5\% HS; (b) 1\% HS.}
\end{figure}

In summary for 5\% quantile forecasting: the VBDQ method was clearly superior for lower sample sizes of $n=250, 500$, and was
quite competitive for $n=1000, 2500$. At these higher sample sizes, the BFDQ4 was marginally favoured, but the DQ4, BFDQ1 and DQ1
all had highly comparable power. The out-performance in size-adjusted power for the VBDQ when $n \le 500$ is attributable to more
accurate detection of autocorrelation in violations from the HS estimator, when the observed violation numbers were comparatively small, compared
to the DQ statistics, and simultaneously higher power than the BFDQ methods for comparatively high violation numbers.
Further, the VBVQ method had higher size-adjusted power than the BFVQ test, except at $n=2500$ and always had higher size-adjusted
power than the VQR test.

Table \ref{size1} shows the empirical estimates for size and then adjusted size across all the methods employed at $\alpha=0.01$ for
$n=250, 500$. Also shown are the empirical $5\%$ points for the methods. Again, empirical size for the BF methods is only reported as a
reference for comparison.

\begin{table}
\begin{center}
\caption{Size, empirical threshold and adjusted size for nominal 5\% tests of 1-step-ahead quantile forecasts at $\alpha=0.01$.
} \label{size1}
\begin{tabular}{lcccccccccccc}
\hline
 & \multicolumn{3}{c}{n=250} & \multicolumn{3}{c}{n=500}      \\
Method   &      Size  &         threshold &   adj. size &      Size  &          threshold &  adj. size    \\ \hline
DQ1      &\fbox{0.052}&             8.827 &      0.0461 &      0.070 &             11.736 &  \cblue{0.0496} \\ [1.3pt]
BFDQ1    &      0.008 &$8.21\times10^{-5}$&      0.0460 &      0.006 &             0.0019 &  \cblue{0.0496} \\ [1.3pt]
VBDQ     &      0.000 &$1.08\times10^{-4}$&      0.0461 &      0.000 &6.53$\times10^{-5}$ &  \cblue{0.0496} \\ [1.3pt]
DQ4      &      0.012 &            26.895 &      0.0461 &      0.168 &             21.144 &  \cblue{0.0496} \\ [1.3pt]
BFDQ4    &      0.011 &$2.14\times10^{-6}$&      0.0459 &      0.008 &1.199$\times10^{-5}$&  \cblue{0.0496} \\ [1.3pt]
VQR      &      0.071 &             7.992 &\fbox{0.0500}&      0.039 &              5.058 &\fbox{0.0500} \\ [1.3pt]
BFVQ     &      0.555 &$2.07\times10^{10}$&\fbox{0.0500}&      0.508 &8.735$\times10^{10}$&\fbox{0.0500} \\ [1.3pt]
VBVQ     &      0.116 &$3.41\times10^{10}$&\fbox{0.0500}&\fbox{0.052}&              1.570 &\fbox{0.0500} \\
\hline
\end{tabular}
\par\smallskip
\parbox{.9\textwidth}{ The correct, nominal thresholds are: VQR 5.99; DQ1 7.81; DQ4 12.59}
\end{center}
\end{table}

At $n=250$, only the DQ1 test achieves close to the true nominal size, with $5.2\%$, the DQ4 is well under-sized and the VQR is well over-sized.
Only the VQR, VBVQ and BFVQ tests achieve correct adjusted sizes of $5\%$; though the others are only marginally under-sized.
For $n=500$, the DQ1 and DQ4 are both over-sized, the VQR is under-sized and no method is close to the nominal level.
Table \ref{size11} shows the empirical estimates for size and then adjusted size across all the methods employed at
$\alpha=0.01$ for $n=1000,2500$. Also shown are the empirical $5\%$ points for the methods across the 25000
replications used to calculate the adjusted size and size-adjusted power.

\begin{table}
\begin{center}
\caption{Size, empirical threshold and adjusted size for nominal 5\% tests of 1-step-ahead quantile forecasts at $\alpha=0.01$.
} \label{size11}
\begin{tabular}{lcccccc}
\hline
 & \multicolumn{3}{c}{n=1000} & \multicolumn{3}{c}{n=2500}                    \\
Method   &       Size  &           threshold &   adj. size &       Size  &           threshold &   adj. size    \\ \hline
DQ1      &\cblue{0.0788}&              9.356 &\fbox{0.0500}&  \cblue{0.0560}&               8.352 &\fbox{0.0500}   \\ [1.3pt]
BFDQ1    &       0.0070 &9.933$\times10^{-5}$&\fbox{0.0500}&      0.0008 &           0.0000115 &\fbox{0.0500}   \\ [1.3pt]
VBDQ     &       0.0000 &3.101$\times10^{-5}$&\fbox{0.0500}&      0.0000 & 1.223$\times10^{-9}$&\fbox{0.0500}   \\ [1.3pt]
DQ4      &       0.1010 &             16.315 &\fbox{0.0500}&      0.0968 &              15.512 &\fbox{0.0500}   \\ [1.3pt]
BFDQ4    &       0.0039 &6.114$\times10^{-7}$&\fbox{0.0500}&      0.0005 & 3.541$\times10^{-9}$&  \cblue{0.0499}   \\ [1.3pt]
VQR      & \fbox{0.0371}&              5.010 &\fbox{0.0500}&\fbox{0.0486}&               5.885 &\fbox{0.0500}   \\ [1.3pt]
BFVQ     &       0.4502 &8.314$\times10^{10}$&\fbox{0.0500}&      0.3248 &  2.966$\times10^{9}$&\fbox{0.0500}   \\ [1.3pt]
VBVQ     &       0.0116 &1.887$\times10^{-8}$&\fbox{0.0500}&      0.0002 &4.379$\times10^{-33}$&\fbox{0.0500}   \\ [1.3pt]
\hline
\end{tabular}
\par\smallskip
\parbox{.9\textwidth}{ The correct, nominal thresholds are: VQ 5.99; DQ1 7.81; DQ4 12.59}
\end{center}
\end{table}

At $n=1000$, again no method is close to the nominal 5\% level: DQ1, DQ4 are over and VQR is under-sized; all methods
have exactly nominal corrected size. For $n=2500$ the VQR test has the closest to nominal size ($4.9\%$), whilst
all methods achieve an adjusted size exactly equal to nominal.

\begin{figure}[thp]
     \centering
      \includegraphics[width=1.0\textwidth]{pdq1000.png}
\caption{\label{pdq1000} Rejection rates for the VBDQ (dark grey diamond), BFDQ1 (medium grey circle), BFDQ4 (light grey circle),
DQ1 (black square) and DQ4 (grey square) statistics against number of violations under the HS quantile estimator at $n=1000$;
(a) 5\% HS; (b) 1\% HS.}
\end{figure}

\begin{figure}%[htp]
     \centering
      \includegraphics[width=0.8\textwidth]{pvq1000.png}
\caption{\label{pvq1000} Rates of rejection for the VBVQ (dark grey diamond), BFVQ (light grey circle) and VQR (black square) statistics against the
number of violations under the HS quantile estimator at $n=1000$; (a) 5\% HS; (b) 1\% HS.}
\end{figure}

Table \ref{power1} shows the empirical estimates for power and size-adjusted power across all the methods
at $\alpha=0.01$. At $n=250$ two method stand out regarding size-adjusted power: VBVQ and BFDQ1;
VBVQ is next best with $\sim 39\%$ and DQ1 is next with $33\%$. To examine power in more detail, consider
Figure \ref{powerDQBF250}(b), showing two times the logarithm of the VBDQ, BFDQ1 and DQ1 statistics, plot against the number of
violations from the incorrect HS 1\% VaR estimator. As is logical, with an expected number of only 2.5 violations under the null,
all methods always reject the HS estimator only for very high numbers of violations (e.g. 7 and above for DQ1, 9 and above for VBDQ).
Gerlach et al (2014) found the out-performance of BFDQ1 is due to a much higher rate of model rejection for low violation numbers,
here 1-4, as shown in Figure \ref{pdqBF250}(b) (comparing VBDQ, BFDQ1 and DQ1 in terms of rejection rates against number of violations),
all of which occur highly frequently at $n=250$ and 1\% HS forecasting. The DQ1 has higher power than BFDQ1 at 5-12 violations,
but these are far less likely to occur in this case. The VBDQ test mimics the higher rejection rates or BFDQ1 for 1-4 violations, and
mimics the higher rejection rates of DQ1 for 5-12 violations, thus ensuring it is significantly higher size-adjusted power than
both DQ1 and BFDQ1 in this case.

\begin{table}
\begin{center}
\caption{Power and size-adjusted power for nominal 5\% tests of 1-step-ahead quantile forecasts at $\alpha=0.01$.
}\label{power1}
\begin{tabular}{lcccccccccccc}
\hline
 & \multicolumn{2}{c}{n=250} & \multicolumn{2}{c}{n=500} & \multicolumn{2}{c}{n=1000} & \multicolumn{2}{c}{n=2500}                   \\
Method   &     Power   &  Size-adj.  &       Power &  Size-adj.  &      Power  &   Size-adj. &       Power &     Size-adj. \\ \hline
DQ1      &\cblue{0.353}&       0.329 &\cblue{0.467}&\cblue{0.367}&\cblue{0.645}&\cblue{0.582}&\cblue{0.931}&\cblue{0.916}  \\ [1.3pt]
BFDQ1    &       0.061 &\cblue{0.459}&       0.055 &       0.179 &       0.058 &       0.379 &       0.100 &       0.823   \\ [1.3pt]
VBDQ     &       0.004 & \fbox{0.505}&       0.003 & \fbox{0.480}&       0.005 &\cblue{0.601}&       0.027 &\cblue{0.904}  \\ [1.3pt]
DQ4      &\cblue{0.399}&       0.288 &\cblue{0.631}&\cblue{0.431}& \fbox{0.700}& \fbox{0.629}&\cblue{0.953}&\cblue{0.916}  \\ [1.3pt]
BFDQ4    &       0.066 &       0.261 &       0.077 &       0.258 &       0.072 &       0.366 &       0.110 &       0.807   \\ [1.3pt]
VQR      &       0.102 &       0.078 &       0.063 &       0.075 &       0.480 &       0.171 &       0.563 &       0.567   \\ [1.3pt]
BFVQ     & \fbox{0.812}&       0.261 & \fbox{0.891}&       0.344 &       0.433 &       0.557 &\cblue{0.995}& \fbox{0.946}  \\ [1.3pt]
VBVQ     &\cblue{0.528}&\cblue{0.389}&\cblue{0.427}&\cblue{0.418}&\cblue{0.582}&\cblue{0.580}&       0.246 &\cblue{0.910}  \\ [1.3pt]
\hline
\end{tabular}
\par\smallskip
\parbox{.9\textwidth}{ The correct, nominal thresholds are: VQ 5.99; DQ1 7.81; DQ4 12.59}
\end{center}
\end{table}

Gerlach et al (2014) found that the performance of the BFDQ1, BFDQ4 methods compared very unfavourably with that of
DQ1, DQ4 for $n>250$ and our results concur with that. However, the VBDQ statistic has no such poor comparison.
For $n=500$, the stand-out method with highest size-adjusted power is again the VBDQ ($48\%$), followed by the
DQ4 ($43\%$) and VBVQ ($42\%$). In this case, the BFDQ1 and BFDQ4 statistics out-performed the DQ1 and DQ4 respectively,
only when 1-4 violations were observed, but these are not very likely when $n=500$ at 1\% forecasting, and the
out-performance of the DQ statistics for the more frequently occurring violation numbers of 5-12 was enough to
overall clearly out-perform both BFDQ statistics here. Again the VBDQ statistic mimicked the higher rejection
rates of the DQ1 and DQ4 statistics for higher numbers of violations, and the higher rejection rates of the BFDQ1 for smaller
numbers of violations, ensuring it was the best method re size-adjusted power overall. Similar results occurred for
$n=1000, 2500$, though the DQ4 statistic had the highest size-adjusted power for $n=1000$, closely followed by the
VBDQ, DQ1 and VBVQ statistics, clearly out-performing their BFDQ counterparts. Figure \ref{pvq1000} plot (b) shows
rejection rates for the VBVQ, BFVQ and VQR methods, for $5\%$ VaR forecasting. Clearly, at almost all violation numbers
both the VBVQ and BFVQ have comparable and much higher rejection rates compared to the VQR test.

However, at $n=2500$ the BFVQ test recorded clearly the highest size-adjusted power with 95\%,
followed by DQ1, DQ4, VBVQ and VBDQ with $\sim 90-92\%$. At all sample sizes except $n=2500$, the VBVQ mmethod had much higher
size-adjusted power than the BFVQ method and always had much higher power than the VQR test.

Overall, for 1\% quantile forecasting, the results clearly favour the VBDQ method for $n \le 500$. At $n=1000$ the DQ4 test is marginally
favoured over the VBDQ method. Finally, at $n=2500$, the BFVQ method is favoured regarding size-adjusted power, followed closely by
the DQ4 and VBVQ methods. Further, the VBVQ method always had higher size-adjusted power than the VQR test.

%\section{Discussion}
%When detecting the incorrect HS estimator of 1\% and 5\% quantiles using a range of competing tests/methods, fairly similar stories can be told at
%each quantile level. First, the VBDQ method is almost always prevalent at or near the top of the rankings regarding size-adjusted power.
%On the contrary, the VQR method always performed towards the bottom on this aspect. The reported performance of the VQR test
%in terms of size-adjusted power is slightly worse than the results in Gaglianone et al. (2011), though
%nearly comparable, but agree with those in Gerlach et all (2014).

%The relatively poor performance of the VQR, compared to the BFVQ, test bears more examination. Figure \ref{VQ1000} shows two times the
%logarithm of the BFVQ (upper) and VQ (lower) test statistics, plot against the number of violations, under the null hypothesis (black circles)
%and also under the HS quantile estimator (grey diamonds), at $n=1000$. Also shown are two times the logarithm of the
%empirical 5\% points of the BFVQ (upper) and VQR (lower) statistics; points above these lines represent rejections of the null hypothesis.
%It is immediately apparent that slightly more violations tend to occur under the HS estimator than under the null, the latter having a distribution
%shifted to the right compared to the former (the ratio of means is 1.34 as mentioned previously). Further, it is clear that the BFVQ
%statistic has a clearly distinguished distribution of values, typically higher, under the HS estimator (grey diamonds) compared to that
%under the null; this leads to the observed 55.7\% size-adjusted power of the BFVQ method. On the contrary, the VQR test statistic does not
%have a clearly distinguished distribution of values under the HS estimator compared to that the null, leading to its very low size adjusted power
%(17.4\%). Similar plots, not shown, occur at $n=250, 500, 2500$.
%
%\begin{figure}%[htp]
%     \centering
%      \includegraphics[width=0.95\textwidth]{VQ1000.png}
%\caption{\label{VQ1000} Two times the logarithm of the (a) BFVQ and (b) VQ test statistics, plot against the number of violations,
%under the null hypothesis (black circles) and the HS quantile estimator (grey triangles) at $n=1000$.
%The horizontal lines are two times the logarithm of the empirical 5\% points of the BFVQ (a) and VQR (b) statistics.}
%\end{figure}

When comparing the VB or BF version of each test with its frequentist counterpart (e.g. VBDQ vs BFDQ1 vs DQ1, etc),
at the 5\% quantile level the results are very clear: at each sample sizes the Bayesian version of the test had
higher size-adjusted power, often only marginally but sometimes much, much higher, than its frequentist competitor, for all the
tests considered (except VQR, BFVQ at $n=1000$). This is a very strong and clear result in favour of the Bayesian method.

For 1\% quantile forecasting, the comparison is not so clear: while the results clearly favour the VBDQ method for
$n \le 500$, at $n=1000$ the DQ4 test is marginally favoured over the VBDQ method. Finally, at $n=2500$, the
BFVQ method is favoured regarding size-adjusted power, followed closely by the DQ4 and VBVQ methods. However, the
VBVQ method always had higher size-adjusted power than the VQR test.

\section{Empirical study}
We briefly report the results of a large empirical study here. Seven daily financial time series: prices, exchange rates or
financial indices, are considered, in each case converting these to daily percentage log returns. The seven series are: the US
S\&P500 index, the AORD, FTSE100 and Hang Seng indices, the AU US exchange rates, the EU US exchange rates and IBM asset prices.
The initial sample period is specifically from January 2, 1998 to December 15, 2005, approximately 2000 days in each case. The forecast
period is December 16, 2005 to January 15, 2010, covering close to 1000 trading days in each market, and including the well-known
global financial crisis (GFC) period.

One-step-ahead forecasts of VaR at 5\% and 1\% quantile levels are estimated under a range of competing models and methods, for each day in
the forecast period. Forecasts for each of four types of heteroskedastic model: the GARCH of Bollerslev (1986), the GJR-GARCH of
Glosten et al (1993) the Threshold (T-)GARCH in Chen, Gerlach and So (2006) and a smooth transition (ST-)GARCH as in Chen and Gerlach (2008)
are estimated employing the MCMC methods of Chen et al. (2012). Each of these specifications is estimated under five types of error
distribution: Gaussian, Student-t, the skewed Student-t of Hansen (1994), the Asymmetric Laplace (AL) of Chen et al (2012) and the
Two-sided Weibull (TW) of Chen and Gerlach (2013). This gives 20 models generating VaR forecasts at 5\% and 1\% quantile levels for 1000 days.
Also considered are the non-parametric 50 day and 250 day sample percentile HS methods, thus giving a total of 22 models or methods.
Estimation results are not shown to save space, since only the test results are directly relevant to this paper; it is expected that most models
and methods will be rejected since the data includes the GFC period, where that outcome is common;
but models and methods that can better captured highly changing volatility and fat-tailed returns will be rejected the least,
across the seven series.

Tables \ref{reject5} and \ref{reject1} show the number of series, out of 7, that each model or method of VaR estimation was rejected in,
using the DQ1, DQ4, BFDQ1, BFDQ4, VBDQ, VQR, BFVQ and VBVQ tests for 5\% and 1\% VaR forecasting, respectively.
The tests were conducted at the 5\% level using the empirical cut-offs in Tables \ref{size5}-\ref{size11} above.

\begin{table}
\begin{center}
\caption{Number of rejections for each model across 7 series for 1\% VaR forecasting.}\label{reject5}
\begin{tabular}{lcccccccc}
\hline
    Method &    DQ1 &    DQ4 &  BFDQ1 &  BFDQ4 &   VBDQ &   VQ   &   BFVQ &   VBVQ   \\ \hline
       G-n &      6 &      5 &\fbox{0}&      1 &      3 &      2 &      4 &      4   \\
     GJR-n &      4 &      5 &\fbox{0}&\fbox{0}&      4 &      4 &      5 &      6   \\
      TG-n &      6 &      5 &      1 &\fbox{0}&      4 &      3 &      5 &      5   \\
      ST-n &      6 &      6 &      1 &\fbox{0}&      5 &      4 &      6 &      6   \\
       G-t &      1 &      3 &      1 &      3 &\fbox{0}&      2 &      1 &      1   \\
     GJR-t &\fbox{0}&\fbox{0}&\fbox{0}&      2 &      1 &      1 &      1 &      2   \\
     TG-t  &\fbox{0}&\fbox{0}&      1 &      2 &      1 &      1 &      2 &      2   \\
      ST-t &      2 &      1 &      1 &      2 &\fbox{0}&      2 &      2 &      2   \\
     G-SKT &      1 &      1 &\fbox{0}&\fbox{0}&\fbox{0}&\fbox{0}&      1 &      1   \\
   GJR-SKT &      1 &      1 &\fbox{0}&\fbox{0}&\fbox{0}&      1 &      1 &      1  \\
    TG-SKT &      1 &      1 &\fbox{0}&      1 &\fbox{0}&\fbox{0}&      2 &      2   \\
    ST-SKT &      1 &      1 &\fbox{0}&\fbox{0}&\fbox{0}&      1 &      2 &      2   \\
      G-AL &      1 &      2 &      2 &      4 &      1 &      3 &      3 &      3   \\
    GJR-AL &      1 &      2 &      2 &      1 &      1 &      5 &\cred{7}&      6   \\
     TG-AL &      1 &      2 &      3 &      2 &      1 &      5 &      5 &      4   \\
     ST-AL &      1 &      2 &      2 &      2 &      1 &      2 &      4 &      4   \\
      G-TW &      1 &\fbox{0}&\fbox{0}&      1 &\fbox{0}&      1 &      1 &      1   \\
    GJR-TW &\fbox{0}&\fbox{0}&      1 &      1 &      2 &      1 &      2 &      2   \\
     TG-TW &      1 &\fbox{0}&\fbox{0}&      2 &\fbox{0}&      1 &      1 &      1   \\
     ST-TW &      1 &\fbox{0}&\fbox{0}&      2 &\fbox{0}&\fbox{0}&      1 &      1   \\
     HS250 &\cred{7}&\cred{7}&      3 &      6 &\cred{7}&      5 &\cred{7}&\cred{7}   \\
      HS50 &\cred{7}&\cred{7}&\cred{7}&      3 &\cred{7}&\cred{7}&\cred{7}&\cred{7}  \\ \hline
\end{tabular}
\end{center}
\end{table}

For 1\% VaR forecasting, the various tests often have quite different numbers of rejections for each model. The DQ tests reject all
models with Gaussian errors in most series, as do the VBDQ, BFVQ and VBVQ tests; however, the BFDQ tests seem to have lower power
at rejecting these, perhaps reflecting their comparatively low power in the simulation study. A similar result occurred for
the HS methods, where the VBDQ and VBVQ also rejected these in all 7 series. Broadly in agreement with Chen et al (2012),
the HS and models with Gaussian errors can be rejected as adequate 1\% VaR forecasters, whilst models with TW and SKT errors are
rejected the least across the various tests, and thus seem most suitable and accurate as 1\% VaR forecasters for these series.


For 5\% VaR forecasting, the results are qualitatively similar. One difference is that models with Gaussian errors are rejected
less frequently for 5\% VaR forecasting; whilst models with AL errors are added to those that are rejected only in a few of the series,
across the various tests.

\begin{table}
\begin{center}
\caption{Number of rejections for each model across 7 series for 5\% VaR forecasting.}\label{reject1}
\begin{tabular}{lccccccccc}
\hline
    Method &    DQ1 &    DQ4 &  BFDQ1 &  BFDQ4 &   VBDQ &   VQ   &   BFVQ &   VBVQ   \\ \hline
       G-n &\fbox{0}&      3 &\fbox{0}&      1 &\fbox{0}&\fbox{0}&\fbox{0}&\fbox{0}  \\
     GJR-n &      2 &      3 &\fbox{0}&      1 &      1 &      1 &      2 &      2   \\
      TG-n &      3 &      3 &      1 &      1 &      1 &      2 &      2 &      2   \\
      ST-n &      3 &      3 &      1 &      1 &      1 &      1 &      2 &      3   \\
       G-t &      1 &      3 &      1 &      3 &      1 &      2 &      2 &      1   \\
     GJR-t &      1 &      2 &      1 &      2 &      1 &      1 &      1 &      1   \\
      TG-t &      1 &      2 &      1 &      2 &      1 &      2 &      1 &      1   \\
      ST-t &      1 &      2 &      1 &      2 &      1 &      3 &      1 &      1   \\
     G-SKT &\fbox{0}&      3 &\fbox{0}&      1 &\fbox{0}&\fbox{0}&\fbox{0}&\fbox{0}  \\
   GJR-SKT &      1 &      3 &\fbox{0}&\fbox{0}&\fbox{0}&      1 &\fbox{0}&\fbox{0}  \\
    TG-SKT &      2 &      3 &\fbox{0}&\fbox{0}&\fbox{0}&      2 &      1 &      3     \\
    ST-SKT &      1 &      3 &\fbox{0}&\fbox{0}&\fbox{0}&      1 &      1 &      2     \\
      G-AL &\fbox{0}&      2 &\fbox{0}&      2 &\fbox{0}&      1 &\fbox{0}&\fbox{0}   \\
    GJR-AL &      1 &      2 &      1 &      1 &\fbox{0}&      1 &      1 &\fbox{0}   \\
     TG-AL &\fbox{0}&      2 &      1 &      1 &\fbox{0}&      1 &      1 &      1     \\
     ST-AL &\fbox{0}&      2 &\fbox{0}&      1 &\fbox{0}&      1 &\fbox{0}&      1     \\
      G-TW &\fbox{0}&      2 &\fbox{0}&      1 &\fbox{0}&\fbox{0}&\fbox{0}&\fbox{0}     \\
    GJR-TW &\fbox{0}&      2 &\fbox{0}&      1 &\fbox{0}&\fbox{0}&\fbox{0}&      1     \\
     TG-TW &\fbox{0}&      2 &\fbox{0}&\fbox{0}&\fbox{0}&\fbox{0}&\fbox{0}&      1     \\
     ST-TW &      1 &      2 &\fbox{0}&\fbox{0}&\fbox{0}&      1 &\fbox{0}&\fbox{0}     \\
     HS250 &\cred{7}&      6 &\cred{7}&\cred{7}&\cred{7}&\cred{7}&\cred{7}&\cred{7}     \\
     HS50  &      6 &      5 &\cred{7}&\cred{7}&      6 &\cred{7}&\cred{7}&\cred{7}   \\  \hline
\end{tabular}
\end{center}
\end{table}



\section{Conclusion}
Two variational Bayes methods are developed for assessing and testing forecast accuracy for dynamic quantile forecasts. The
tests are analogues of the VQR and DQ tests commonly applied in this area. The proposed methods do not depend on the
estimation technique, nor on the model or method used, to generate the quantile forecasts.
In a simulation study, at both $\alpha=0.01, 0.05$ quantile levels, these methods performed favourably, regarding
size-adjusted power, compared to their competing Bayesian and frequentist analogues.

\vspace{3cm}


\noindent{\bf Acknowledgement}
The authors thank Declan Walpole for coding up the VQR test.

\newpage
\noindent{\bf References}

\begin{description}
\item
Berkowitz, J., Christoffersen, P., and Pelletier, D. (2011), ``Evaluating Value-at-Risk Models with Desk-Level Data,''
{\em Management Science}, {\bf 57}, 2213-2227.

\item Bollerslev, T. (1986). ``Generalized Autoregressive Conditional Heteroskedasticity,'' {\em Journal of Econometrics},
{\bf 31}, 307-327.

\item Brown, L. D., Cai, T. T., and DasGupta, A. (2001) ``Interval Estimation for a Binomial Proportion,''
{\em Statistical Science}, {\bf 16}, 101-117.

\item Chen, C. W. S., Gerlach, R., Hwang, R. B. K., and McAleer, M. (2012), ``Forecasting Value-at-Risk
using nonlinear regression quantiles and the intra-day range,'' {\em International Journal of Forecasting}, {\bf 28}, 557-574.

\item
Chen, C. W. S., Gerlach, R., Lin, E. M. H., and Lee, W.C.W. (2012), ``Bayesian forecasting for financial risk management, pre and post the
global financial crisis,'' {\em Journal of Forecasting}, {\bf 31}, 661-687.

\item
Chen, C. W. S., and  So, M. K. P. (2006). ``On a threshold heteroscedastic model,'' {\em International Journal of Forecasting}, {\bf 22},
73-89.

\item
Chen, Q., and Gerlach, R. (2013). ``The two-sided Weibull distribution and forecasting financial tail risk,''
{\em International Journal of Forecasting}, {\bf 29}, 527-540.

\item
Chen, Q., Gerlach, R. and Lu, Z. (2012), ``Bayesian Value-at-Risk and expected shortfall forecasting via the asymmetric
Laplace distribution,'' {\em Computational Statistics \& Data Analysis}, 1st Issue of Annals of Computational and Financial
Econometrics, {\bf 56}, 3498-3516.

\item Christoffersen, P. F. (1998), ``Evaluating interval forecasts,'' {\em International Economic Review}, {\bf39}, 841-862.

\item Engle R. F., and Manganelli S. (2004), ``CAViaR: Conditional autoregressive value at risk by regression quantiles,''
{\em Journal of Business and Economic Statistic}, {\bf 22}, 367- 381.

\item Gaglianone, W. P., Lima, L. R., Linton, O., and Smith, D. R. (2011), ``Evaluating Value-at-Risk models via quantile regression,''
    {\em Journal of Business \& Economic Statistics}, {\bf 29}, 150-160.

\item
Gelman, A., Carlin, J. B., Stern, H. S., and Rubin, D. B. (2005), Bayesian Data Analysis (2nd ed.), Boca Raton, FL: Chapman \& Hall.

\item Gerlach, R., and Chen, C. W. S. (2008), ``Bayesian inference and model comparison for asymmetric smooth transition
   heteroskedastic models,'' {\em Statistics and Computing}, {\bf 18}, 391-408.

\item  Gerlach, R., Chen, C.W.S and Lin EMH (2014), ``Bayesian Assessment of Dynamic Quantile Forecasts'', The University of
    Sydney Business Analytics Working Paper Series, http://sydney.edu.au/business/business_analytics/research/working_papers.

\item
Girolami, M. and Rogers, S. (2006) ``Variational Bayesian multinomial probit regression with Gaussian process priors,''
{\em Neural Computation} 18, {\bf 8}, 1790-1817.

\item Glosten, L. R., Jagannathan, R., and Runkle, D. E. (1993). ``On the relation between the expected value and the volatility
of the nominal excess return on stocks,'' {\em Journal of Finance}, {\bf 487}, 1779-1801.

\item Hansen, B., (1994). ``Autoregressive conditional density estimation,'' {\em International Economic Review}. {\bf 35}, 705–730.

\item Hoogerheide, L. F., and van Dijk, H.K. (2010), ``Bayesian forecasting of Value at Risk and expected shortfall using
adaptive importance sampling", {\em International Journal of Forecasting}, {\bf 26}, 231-247.

\item
Koenker, R., and Machado, J. A. F. (1999) ``Goodness of fit and related inference for quantile regression,'' {\em Journal of the American
Statistical Association}, {\bf 94}, 1296-1310.

\item
Kupiec, P. (1995) ``Techniques for verifying the accuracy of risk measurement models,'' {\em Journal of Derivatives},
{\bf 2}, 173-84.

\item
Ormerod, J. and Wand, M. (2010) ``Explaining variational approximations,'' {\em American Statistician}, 64, {\bf 2}, 140-153.

\item
Smith, M., and Kohn, R. (1996). ``Non-parametric Regression Using Bayesian Variable Selection,'' {\em Journal of Econometrics},
{\bf 75}, 317-343.

\item
Tzikas, D., Likas, A.  and Galatsanos, N. ``Variational Bayesian sparse kernel-based blind image deconvolution with student's-t priors,'' IEEE Transactions on Image Processing, vol. 18, no. 4, pp. 753-764.

\item Tuyl, F., and Gerlach, R., and Mengersen, K. (2008). ``Inference for Proportions in a 2 x 2 Contingency Table: HPD or not HPD?,''
{\em Biometrics}, {\bf 64}, 1293-1296.



\end{description}

\clearpage
\begin{appendices}


\section{Approximate Posteriors for Quantile Regression Case}
This section on deriving approximate posteriors on the latent variables for quantile regression is a novel contribution. The Bayesian hierarchy required for this section is shown in \eqref{BFVQ_hier}.

\subsection{ $q(\beta)$}
\label{sec:q_beta}
Approximate posterior on $\cbeta$ and relevant expectations
\begin{align}
\nonumber
\log q(\cbeta)&=\clrangle{-\chalf{}\frac{\caexp}{2\sigma^2}\csum\frac{1}{w_i}\clrbracket{\cbeta^T \cW_t+\frac{1-2\alpha}{\caexp}\sigma w_t-y_t}^2}_{q(w_t)q(\sigma)}\\
\nonumber
&\qquad\qquad\qquad\qquad\qquad\qquad\qquad\qquad\qquad\qquad-\chalf{}\times\frac{1}{1000}\cbeta^T\cbeta\\
\nonumber
&=-\chalf{}\Biggl[\cbeta^T\clrbracket{\frac{\caexp}{2}\clrangle{\frac{1}{\sigma^2}}\csum\clrangle{\frac{1}{w_t}}\cW_t\cW_t^T+\frac{1}{1000}\cI}\cbeta\\
\nonumber
&-2\clrbracket{\csum\left[\frac{\caexp}{2}\clrangle{\frac{1}{\sigma^2}}\clrangle{\frac{1}{w_t}}y_t-\frac{1-2\alpha}{2}\clrangle{\frac{1}{\sigma}}\right]\cW_t^T}\cbeta\Biggr]\\
\therefore q(\cbeta)&\sim\cN(\cmu,\cSigma)\\
\cSigma &=\clrbracket{\frac{\caexp}{2}\clrangle{\frac{1}{\sigma^2}}\csum\clrangle{\frac{1}{w_t}}\cW_t\cW_t^T+\frac{1}{1000}\cI}^{-1}\\
\cmu&=\cSigma\clrbracket{\csum\clrbracket{\frac{\caexp}{2}\clrangle{\frac{1}{\sigma^2}}\clrangle{\frac{1}{w_t}}y_t-\frac{1-2\alpha}{2}\clrangle{\frac{1}{\sigma}}}\cW_t}\\
\clrangle{\cbeta}&=\cmu\\
\clrangle{\cbeta\cbeta^T}&=\cSigma+\cmu\cmu^T
\end{align}

\subsection{ $q(\cw)$}
Approximate posterior on $w_i$ and relevant expectations.
\begin{align}
\nonumber
\log (q(w_i))&=-w_i-\chalf{}\log(w_i)\\
\nonumber
&\qquad\qquad\quad-\chalf{}\clrangle{\frac{\caexp}{2\sigma^2 w_i}\clrbracket{\cbeta^T \cW_i+\frac{1-2\alpha}{\caexp}\sigma w_i-y_i}^2}_{q(\cbeta)q(\sigma)}\\
\nonumber
&=-w_i-\chalf{}\log(w_i)\\
\nonumber&-\chalf{}\clrbracket{\frac{(1-2\alpha)^2}{2\caexp}w_i+\frac{\caexp}{2}\clrangle{\frac{1}{\sigma^2}}\clrbracket{y_i^2-2y_i\cW_i^T\clrangle{\cbeta}+\cW_i^T\clrangle{\cbeta\cbeta^T}\cW_i}\frac{1}{w_i}}\\
\nonumber
&=-\chalf{}\log(w_i)\\
\nonumber&-\chalf{}\clrbracket{\clrbracket{\frac{(1-2\alpha)^2}{2\caexp}+2}w_i+\frac{\caexp}{2}\clrangle{\frac{1}{\sigma^2}}\clrbracket{y_i^2-2y_i\cW_i^T\clrangle{\cbeta}+\cW_i^T\clrangle{\cbeta\cbeta^T}\cW_i}\frac{1}{w_i}}\\
\therefore q(w_i) & = \cGIG\clrbracket{\chalf{},\alpha,\beta}\\
\alpha_i & =\clrbracket{\frac{(1-2\alpha)^2}{2\caexp}+2} \\
\beta_i & = \frac{\caexp}{2}\clrangle{\frac{1}{\sigma^2}}\clrbracket{y_i^2-2y_i\cW_i^T\clrangle{\cbeta}+\cW_i^T\clrangle{\cbeta\cbeta^T}\cW_i}\\
\clrangle{\frac{1}{w_i}}&=\sqrt{\frac{\alpha_i}{\beta_i}}\\
\clrangle{w_i}&=\sqrt{\frac{\beta_i}{\alpha_i}}+\frac{1}{\alpha_i}
\end{align}

\subsection{$q(\sigma)$}
\label{sec:q_sigma}
Approximate posterior on $\sigma$ and relevant expectations.
\begin{align}
\nonumber \log q(\sigma) & =-\frac{N}{2}\log \sigma^2-\chalf{}\csum\clrangle{\frac{\caexp}{2 w_i}\clrbracket{\cbeta^T \cW_i+\frac{1-2\alpha}{\caexp}\sigma w_i-y_i}^2}_{q(\cbeta)q(w_i)}\frac{1}{\sigma^2}\\
\nonumber &\qquad\qquad\qquad\qquad\qquad\qquad\qquad\qquad\qquad\qquad\qquad\qquad-\log \sigma\\
\nonumber &=-(N+1)\log \sigma-\gamma\frac{1}{\sigma}-\delta\frac{1}{\sigma^2}\\
\gamma &= -{\frac{1-2\alpha}{2}\csum(y_i-\cW_i^T\clrangle{\cbeta})}\\
\delta &=\frac{\caexp}{4}\csum\clrangle{\frac{1}{w_i}}\clrbracket{y_i^2-2y_i\cW_i^T\clrangle{\cbeta}+\cW_i^T\clrangle{\cbeta\cbeta^T}\cW_i}\\
\clrangle{\frac{1}{\sigma}}&=\frac{N}{\sqrt{2\delta}}\,\frac{U(N+\chalf{},\frac{\gamma}{\sqrt{2\delta}})}{U(N-\chalf{},\frac{\gamma}{\sqrt{2\delta}})}\\
\clrangle{\frac{1}{\css}}&=\frac{N(N+1)}{2\delta}\,\frac{U(N+\frac{3}{2},\frac{\gamma}{\sqrt{2\delta}})}{U(N-\chalf{},\frac{\gamma}{\sqrt{2\delta}})}
\end{align}

However, the above approximate posterior on $q(\sigma)$ suffers from numerical problems due to calculations of the parabolic cylindrical function. Hence, we shall restrict $q(\cbeta)=IG(a,b)$. VB requires the maximisation of the lower bound $F(q)$ which can be expressed as $-KL(q_j||\tilde{p})-\sum_{i\ne j}\int q_i\log q_i dz$ where $\log\tilde{p}=\int \log p(\cy,\cz)\prod_{i\ne j}(q_i dz_i)$. Thus we are required to maximise,

\begin{align}
\nonumber F_\sigma=&=-(N+1)\clrangle{\log \sigma}-\gamma\clrangle{\frac{1}{\sigma}}-\delta\clrangle{\frac{1}{\sigma^2}}-\int q(\sigma)\log q(\sigma)\,d\sigma\\
\nonumber \clrangle{\frac{1}{\sigma}}&=\frac{a}{b} \qquad\qquad \clrangle{\frac{1}{\css}}=\frac{a(a+1)}{b^2} \qquad \clrangle{\log \sigma}=\log b -\psi(a)\\
\therefore F_\sigma&=(a-N)(\log b-\psi(a))+(b-\gamma)\frac{a}{b}-\delta \frac{a(a+1)}{b^2}-a\log b+\log \Gamma(a)\\
\frac{\partial F}{\partial a}&=(N-a)\psi^{(1)}(a)-\frac{\gamma}{b}-\frac{\delta(2a+1)}{b^2}+1\\
\frac{\partial F}{\partial b}&=-\frac{N}{b}+\frac{\gamma a}{b^2}+\frac{2\delta a(a+1)}{b^3}
\end{align}
Thus $F_\sigma$ is maximised using a numerical optimiser which employs the given derivatives.

\subsection{Lower Bound for Log Likelihood}
\label{sec:l_bound}
Using the EM algorithm it can be shown that $\int q(\cz)\log p(\cy,\cz)-q(\cz)\log q(\cz)\,d\cz$ is a lower bound to the log data likelihood where, $\cz$ are the latent variables. The first term $\int q(\cz)\log p(\cy,\cz)d\cz$ can be shown to be,
\begin{align}
\nonumber
&=-\chalf{}\Biggl[N\log\frac{4\pi}{\caexp}+\csum\clrangle{\log w_i} +2(N+1)\clrangle{\log\sigma}+D\log (2\pi\times1000)\\
\nonumber&+\clrangle{\cbeta^T\clrbracket{\frac{\caexp}{2}\clrangle{\frac{1}{\sigma^2}}\csum\clrangle{\frac{1}{w_i}}\cW_i\cW_i^T+\frac{1}{1000}\cI}\cbeta}\\
\nonumber
&-2\clrbracket{\csum\clrbracket{\frac{\caexp}{2}\clrangle{\frac{1}{\sigma^2}}\clrangle{\frac{1}{w_i}}y_i-\frac{1-2\alpha}{2}\clrangle{\frac{1}{\sigma}}}\cW_i^T}\clrangle{\cbeta}\\
\nonumber
&+\frac{\caexp}{2}\clrangle{\frac{1}{\css}}\csum\clrangle{\frac{1}{w_i}}y_i^2+\clrbracket{\frac{(1-2\alpha)^2}{2\caexp}+2}\csum\clrangle{w_i}-(1-2\alpha)\clrangle{\frac{1}{\sigma}}\csum y_i\Biggr]\\
\nonumber&=-\chalf{}\Biggl[N\log\frac{4\pi}{\caexp}+\csum\clrangle{\log w_i} +2(N+1)(\log b -\psi(a))+D\log (2\pi\times1000)\\
\nonumber&+D-\cmu^T\cSigma^{-1}\cmu+\frac{\caexp}{2}\frac{a(a+1)}{b^2}\sum_{i=1}^{N}\sqrt{\frac{\alpha_i}{\beta_i}}y_i^2+\clrbracket{\frac{(1-2\alpha)^2}{2\caexp}+2}\csum\clrbracket{\sqrt{\frac{\beta_i}{\alpha_i}}+\frac{1}{\alpha_i}}\\
&-(1-2\alpha)\frac{a}{b}\csum y_i\Biggr]
\end{align}

The second term $\int q(z)\log q(z)dz$ can be shown to be,
\begin{align}
\nonumber&-\chalf{}\Biggl[D\log(2\pi)+\log |\cSigma|+\clrangle{\cbeta^T\cSigma^{-1}\cbeta}-2\cmu^T\cSigma^{-1}\clrangle{\cbeta}+\cmu^T\cSigma^{-1}\cmu\Biggr]\\
&\nonumber+\frac{1}{4}\csum(\log\alpha_i-\log\beta_i)-N\log 2- \csum\log\cBess_{1/2}\clrbracket{\sqrt{\alpha_i\beta_i}}\\
\nonumber&-\chalf{}\csum\Biggl[\clrangle{\log w_i} +\alpha_i\clrangle{w_i}+\beta_i\clrangle{\frac{1}{w_i}}\Biggr]\\
%\nonumber&+\frac{N-2}{2}\log 2\delta-\log U\clrbracket{N-\frac{5}{2},\frac{\gamma}{\sqrt{2\delta}}}-\log \Gamma(N-2)-\clrbracket{\frac{\gamma^2}{8\delta}}\\
%\nonumber&-(N+1)\clrangle{\log\sigma}-\gamma\clrangle{\frac{1}{\sigma}}-\delta\clrangle{\frac{1}{\css}}\\
&\nonumber +a\log b - \log \Gamma(a)-(a+1)\clrangle{\log \sigma}-b\clrangle{\frac{1}{\sigma}}
\\ \nonumber=&-\chalf{}\clrbracket{D+D\log2\pi+\log|\cSigma|}-\frac{N}{2}\log 2-\frac{N}{2}\log\pi+\chalf{}\csum\log\alpha_i+\sqrt{\alpha_i\beta_i}\\
\nonumber&-\chalf{}\csum\Biggl[\clrangle{\log w_i} +2\sqrt{\alpha_i\beta_i}+1\Biggr]+a\log b -\log \Gamma(a)-(a+1)(\log b-\psi(a))-a\\
&\nonumber=-\chalf{}\clrbracket{D+D\log2\pi+\log|\cSigma|}-\frac{N}{2}\log 2-\frac{N}{2}\log\pi+\chalf{}\csum\log\alpha_i-\frac{N}{2}\\
&-\chalf{}\csum\Biggl[\clrangle{\log w_i}\Biggr]-\log \Gamma(a)-\log b+a\psi(a)+\psi(a)-a
%\nonumber&+\frac{N-2}{2}\log 2\delta-\log U\clrbracket{N-\frac{5}{2},\frac{\gamma}{\sqrt{2\delta}}}-\log \Gamma(N-2)-\clrbracket{\frac{\gamma^2}{8\delta}}\\
%\nonumber&-(N+1)\clrangle{\log\sigma}-\gamma\clrangle{\frac{1}{\sigma}}-\delta\clrangle{\frac{1}{\css}}
\end{align}

\section{Approximate Posteriors for Classification Case}
This is an adaptation of the work done by Girolami and Rogers (2006). The Bayesian hierarchy is as follows:
\begin{align}
\nonumber p(\cI_t|\cW_t,\cbeta)&=\Phi^{\cI_t}(\cW_t^T\cbeta)(1-\Phi(\cW_t^T\cbeta))^{1-\cI_t}\\
\nonumber p(\cbeta) & = \cN(0,1000\cI)
\end{align}
The Bernoulli distribution over $\cI_t$ can be restated using a latent variable $y_t$,
\begin{align}
\nonumber p(\cI_t=1|\cW_t,\cbeta)&=\int \delta(y_t>0)\cN(y_t|\cW_t\cbeta,1_)\,dy_t\\
\nonumber p(\cI_t=0|\cW_t,\cbeta)&=\int \delta(y_t<0)\cN(y_t|\cW_t\cbeta,1_)\,dy_t\\
p(\cI_t,y_t|\cW_t,\cbeta)&=\delta(y_t>0)^{\cI_t}\delta(y_t<0)^{1-\cI_t}\cN(y_t|\cW_t\cbeta,1)
\end{align}

\subsection{$q(\beta)$}
\label{sec:class_beta}
Approximate posterior on $q(\beta)$ and relevant expectations,
\begin{align}
\nonumber\log q(\cbeta)&=\log \cN(\cbeta|\czero,1000\cI)-\chalf{}\sum_{t=1}^{N}(\cy_t-\cW_t^T\cbeta)^2\\
\nonumber&=-\chalf{}\clrbracket{\frac{\cbeta^T\cbeta}{1000}+\cbeta^T\cW\cW^T\cbeta-2\clrangle{\cy}^T\cW^T\cbeta}+const\\
\therefore q(\cbeta)&=\cN(\cmu,\cSigma) \label{eq:qb}\\
\cSigma&=\clrbracket{\cW\cW^T+\frac{1}{1000}\cI}^{-1}\\
\cmu&=\cSigma\cW\clrangle{\cy}
\end{align}

\subsection{$q(\cy_t)$}
\label{sec:class_y}
\begin{align}
\nonumber\log q(y_t)&=\cI_t\log \delta(y_t>0)+(1-\cI_t)\log \delta(y_t<0)-\chalf{}(y_t-2y_t\cW_t^T\clrangle{\cbeta})+const\\
q(y_t)&=\cN_{\cI_t}(y_i|\cW_t^T\cmu,1)
\end{align}
Where $\cN_{\cI_t}$ is a left/right truncated normal distribution depending on ${\cI_t}=0$, ${\cI_t}=1$ respectively. The required expectations, $E(y_t)$, $E(y_t^2)$ are shown below. Due to the iterative nature of VB expectations of $q(\cy)$ is required to obtain the posterior on $q(\beta)$. Therefore, the normalising constant, the mean $E(\cy_t)$ as well as $E(\cy_t^2)$ (which can be used to find the variance) is shown below,

\begin{align}
\nonumber C_t&=\begin{cases}
\int_{-\infty}^{0}\exp(-\chalf{}(y_t-\cW_t^T\cmu)^2)\,dy_t & \cI_t=0\\
\int_{0}^{\infty}\exp(-\chalf{}(y_t-\cW_t^T\cmu)^2)\,dy_t & \cI_t=1\\
\end{cases}\\
&=\begin{cases}
\sqrt{2\pi}\Phi(-\cW_t^T\cmu) & \cI_t=0\\
\sqrt{2\pi}(1-\Phi(-\cW_t^T\cmu)) & \cI_t=1\\
\end{cases}\\
\nonumber E(y_t)&=\begin{cases}
\int_{-\infty}^{0}y_t\frac{1}{C_t}\exp\clrbracket{-\chalf{}(\cy-\cW_t^T\cmu)^2}\,dy_t & \cI_t=0\\
\int_{0}^{\infty}y_t\frac{1}{C_t}\exp\clrbracket{-\chalf{}(\cy-\cW_t^T\cmu)^2}\,dy_t & \cI_t=1
\end{cases}\\
&=\begin{cases}
-\frac{1}{C_t}\exp\clrbracket{-\chalf{}(\cW_t^T\cmu)^2}+\cW_t^T\cmu & \cI_t=0\\
\frac{1}{C_t}\exp\clrbracket{-\chalf{}(\cW_t^T\cmu)^2}+\cW_t^T\cmu & \cI_t=1
\end{cases}\\
\nonumber E(y_t^2)&=\begin{cases}
-\frac{\cW_t^T\cmu\exp\clrbracket{-\chalf{}(\cW_t^T\cmu)^2}}{C_t}+(\cW_t^T\cmu)^2+1 & \cI_t=0\\
\frac{\cW_t^T\cmu\exp\clrbracket{-\chalf{}(\cW_t^T\cmu)^2}}{C_t}+(\cW_t^T\cmu)^2+1 & \cI_t=1
\end{cases}
\end{align}

The method in which $E(\cy_t^2)$ is obtained for a general truncated Gaussian is shown below. 

For $\cI_t=1$.
\begin{align}
\nonumber E(\cy_t^2)&=\int_{0}^{\infty}\cy_t^2\frac{1}{C_t}\exp\clrbracket{-\chalf{}(\cy_t-\mu)^2}\,dy \\
\nonumber \text{let }z=\cy_t-\mu \qquad E(\cy_t^2)&=\frac{1}{C_t}\int_{-\mu}^{\infty}z^2\exp\clrbracket{-\chalf{}z^2}dz+2\mu E(\cy_t)-\mu^2\\
\nonumber \text{let } u=z\qquad dv=z\exp\clrbracket{-\chalf{}z^2} &\qquad \therefore du=dz \qquad v=-\exp\clrbracket{-\chalf{}z^2}\\
\nonumber \frac{1}{C_t}\int_{-\mu}^{\infty}z^2\exp\clrbracket{-\chalf{}z^2}dz &=\frac{-z\exp\clrbracket{-\chalf{}z^2}}{C_t}\Bigg|_{-\mu}^\infty+\frac{1}{C_t}\int_{-\mu}^{\infty}\exp\clrbracket{-\chalf{}z^2}\,dz\\
\nonumber &=-\frac{\mu\exp\clrbracket{-\chalf{}\mu^2}}{C_t}+1\\
\nonumber E(\cy_t^2)&=-\frac{\mu\exp\clrbracket{-\chalf{}\mu^2}}{C_t}+1+\mu^2+\frac{2\mu\exp\clrbracket{-\chalf{}\mu^2}}{C_t}\\
&=\frac{\mu\exp\clrbracket{-\chalf{}\mu^2}}{C_t}+\mu^2+1
\end{align}
Similarly for $\cI_t=0$,
\begin{align}
\nonumber E(\cy_t^2)&=\frac{\mu\exp\clrbracket{-\chalf{}\mu^2}}{C_t}+1+2\mu E(\cy_t)-\mu^2\\
&=-\frac{\mu\exp\clrbracket{-\chalf{}\mu^2}}{C_t}+\mu^2+1
\end{align}



\subsection{Lower Bound for Log Likelihood}
\label{sec:class_lik}
In this section a lower bound for the log likelihood is obtained as was done for the quantile regression case using the approximate distributions. Using the EM algorithm, the term, $\int q(\cy)q(\beta)(\log p(\cy,\beta,\cI)-\log q(\cy)q(\beta)) d\beta d\cy$ forms a lower bound to the log likelihood. Let $\delta_{\cI_t}(y_t)=\delta(y_t>0)^{\cI_t}\delta(y_t<0)^{1-\cI_t}$.
\begin{align}
\nonumber \clrangle{\log p(\cy,\cbeta,\ct)}=&\csum\log\delta_{\cI_t}(y_t)\cN(y_t|\cW_t^T\cbeta,1)-\frac{d}{2}\log(2\pi\times1000)-\chalf{}\times\frac{1}{1000}\clrangle{\cbeta^T\cbeta}\\
\nonumber =&-\frac{N}{2}\log(2\pi)-\chalf{}\csum\clrbracket{ E(y_t^2)-2E(y_t)\cW_t^T\cmu+\cW_t^T\clrangle{\cbeta\cbeta^T}\cW_t}\\
\nonumber &-\frac{d}{2}\log(2\pi\times1000)-\chalf{}\times\frac{1}{1000}(\cmu^T\cmu+Tr(\cSigma))\\
\nonumber=&-\frac{N+d}{2}\log 2\pi-\frac{d}{2}\log 1000-\chalf{}\times\frac{1}{1000}(\cmu^T\cmu+Tr(\cSigma))\\
&-\chalf{}\csum\clrbracket{(-1)^{\cI_t}\frac{\cW_t^T\cmu\exp(-\chalf{}(\cW_t^T\cmu)^2)}{C_t}+1+\cW_t^T\cSigma\cW_t}\\
\nonumber\clrangle{\log q(\cbeta)}=&-\chalf{}\log 2\pi|\cSigma|-\chalf{}(\cbeta-\cmu)^T\cSigma^{-1}(\cbeta-\cmu)\\
&=-\frac{d}{2}\log(2\pi)-\chalf{}\log|\cSigma|-\frac{d}{2}\\
\nonumber\clrangle{\log q(y_t)}=&-\log C_t-\chalf{}\clrangle{(y_t^2-2\cW_t^T\cmu y_t+(\cW_t^T\cmu)^2)}\\
=&-\log C_t-\chalf{}\clrbracket{(-1)^{\cI_t}\frac{\cW_t^T\cmu\exp(-\chalf{}(\cW_t^T\cmu)^2)}{C_t}+1}
\end{align}
Thus, the lower bound is as follows:
\begin{align}
=&-\chalf{}\clrbracket{N\log 2\pi+d\log 1000+\frac{\cmu^T\cmu+Tr(\cSigma)}{1000}+\sum_{t=1}^{N}\cW_t^T\cSigma\cW_t-d-\log|\cSigma|-\sum_{t=1}^{N}\log C_t}
\end{align}
\end{appendices}

\end{document}
