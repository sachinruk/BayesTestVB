\documentclass[12pt,epsf]{article}
\usepackage{indentfirst}
\usepackage{array}

\usepackage[usenames]{color}
\usepackage{psfrag,graphicx}
\usepackage{eso-pic,graphicx}
\usepackage{colortbl}

%Sachins packages
\newcommand{\cN}{{\mathcal{N}}}
\newcommand{\cAL}{{\mathcal{AL}}}
\newcommand{\cH}{{\mathcal{H}}}
\newcommand{\cGa}{{\mathbf{Ga}}}

\newcommand{\cR}{{\mathcal{R}^\mathcal{D}}}
\newcommand{\cf}{{\mathbf{f}}}
\newcommand{\cfs}{\mathbf{f_\star}}
\newcommand{\cx}{\mathbf{x}}
\newcommand{\cX}{\mathbf{X}}
\newcommand{\cxs}{\mathbf{x}_\star}
\newcommand{\ct}{{\mathbf{t}}}
\newcommand{\ceps}{{\mathbf{\epsilon}}}
\newcommand{\cI}{\mathbf{I}}
\newcommand{\cA}{\mathbf{A}}
%\newcommand{\cx}{{\mathbf{x}}}
\newcommand{\cB}{\mathbf{B}}
\newcommand{\cD}{\mathbf{D}}
\newcommand{\czero}{{\mathbf{0}}}
\newcommand{\cK}{{\mathbf{K}}}
\newcommand{\cSigma}{{\mathbf{\Sigma}}}
\newcommand{\cem}{{\mathbf{\m}}}
\newcommand{\cmu}{{\boldsymbol{\mu}}}
\newcommand{\cexp}{{\text{exp}}}
\newcommand{\csI}{\sigma^2\cI}
\newcommand{\cts}{\cf}
\newcommand{\cData}{\mathcal{D}}
\newcommand{\cPhi}{{\mathbf{\Phi}}}
\newcommand{\Pzun}{p(z)}
\newcommand{\cpt}{\int_0^{\infty}\Pzun p(\ct|z)\,dz}
\newcommand{\cptb}{\int_0^{\infty}\Pzun~\cN(\ct| \czero,z\cK_{..})\,dz}
\newcommand{\cBess}{\mathcal{K}}
\newcommand{\cBessI}{\mathcal{I}}
\newcommand{\cint}{{\int_0^\infty}}
\newcommand{\cfKinvf}{\cf^T\cK^{-1}\cf}
\newcommand{\cGIG}{\mathbf{GIG}}

\newcommand{\GIG}{{\mathbf{GIG}}}
\newcommand{\cEz}{\left\langle\frac{1}{z}\right\rangle_{q(z)}}
\newcommand{\cEzz}{\left\langle\frac{1}{z}\right\rangle}
\newcommand{\cEfKinvf}{\left\langle\cfKinvf\right\rangle_{q(\cf)}}
\newcommand{\czKsI}{\frac{1}{\cEzz_{q(z)}}\cK+\csI}
%angles paranthesis
\newcommand{\clangle}{\left\langle}
\newcommand{\crangle}{\right\rangle}
\newcommand{\chalf}[1]{\frac{1}{2#1}}
\newcommand{\css}{\sigma^2}
\newcommand{\clrbracket}[1]{\left(#1\right)}
\newcommand{\clrangle}[1]{\clangle#1\crangle}

\newcommand{\ctKinvt}{\ct^T\cK_{..}^{-1}\ct}
\newcommand{\cp}{\tilde{p}}
\newcommand{\ca}{\tilde{\alpha}}
\newcommand{\cb}{\tilde{\beta}}
\newcommand{\cabt}{\ca\cb}

\newcommand{\cU}{\mathbf{U}}
\newcommand{\czKcsI}{z \cK+\csI}
\newcommand{\cKx}{\cK_{\cx,\cx}}
\newcommand{\cv}{\mathbf{v}}

\newcommand{\cS}{\mathbf{S}}
\newcommand{\cw}{\mathbf{w}}
\newcommand{\cz}{\mathbf{z}}
\newcommand{\cW}{\mathbf{W}}
\newcommand{\cy}{\mathbf{y}}
\newcommand{\calpha}{\bm{\alpha}}
\newcommand{\clambda}{\bm{\lambda}}
\newcommand{\cgamma}{\bm{\gamma}}
\newcommand{\ceta}{\bm{\eta}}
\newcommand{\cbeta}{\bm{\beta}}

\newcommand{\caexp}{\alpha(1-\alpha)}
\newcommand{\csum}{\sum_{i=1}^{N}}
\usepackage{amsmath}
\usepackage{amssymb}
\usepackage{bm}
\usepackage{natbib}
\usepackage{hyperref}
\usepackage{breqn}
\usepackage{refstyle}
\usepackage[toc,page]{appendix}
%\usepackage{graphicx}
%\DeclareGraphicsRule{*}{eps}{.bb}{}
\renewcommand{\baselinestretch}{1.5}
\setlength{\oddsidemargin}{0in} \setlength{\evensidemargin}{0in} \setlength{\textwidth}{6.5in}
\setlength{\textheight}{8.3in} \setlength{\headsep}{0in} \setlength{\parindent}{0.3in} \marginparwidth 0pt
\oddsidemargin 0pt \evensidemargin 0pt \marginparsep 0pt \topmargin 0pt \headheight 0pt \headsep 0pt \textheight
23.5 truecm \textwidth 16.0 truecm
\newtheorem{theorem}{Theorem}
\newcommand{\utwi}[1]{\mbox{\boldmath $ #1$}}
\newcommand{\ratio}{($\sigma_{\phi_i}/\tau_{i}, c_i$)}
\newcommand{\hs}{\hspace{1cm}}
\newcommand{\utheta}{\utwi{\theta}}
\newcommand{\ualp}{\utwi{\alpha}}
\newcommand{\calF}{{\cal F}}
\newcommand{\blue}[1]{\textcolor{blue}{$#1$}}
\newcommand{\red}[1]{\textcolor{red}{$#1$}}
\newcommand{\ignore}[1]{}{}
\newcommand{\cblue}{\textcolor{blue}}
\newcommand{\cred}{\textcolor{red}}
\definecolor{slight}{gray}{0.75}
\fboxsep=1pt
\def\r{\color{red}}
\def\cblue{\color{blue}}
\date{}


\begin{document}
%\title{\Large \bf Variational Bayes for Assessment of Dynamic Quantile Forecasts}
%
%\author{
%\normalsize  Richard Gerlach$^{1}$\thanks{Corresponding author: Richard Gerlach, Email: richard.gerlach@sydney.edu.au}
%and Sachin Abeywadena$^{2}$ \\
%\normalsize $^{1}$ The University of Sydney Business School, Australia. \\
%\normalsize $^{2}$ School of Engineering, The University of Sydney, Australia.}
%
%\maketitle

\noindent

\thispagestyle{empty}
\begin{abstract}
Methods for Bayesian testing and assessment of dynamic quantile forecasts have recently been proposed. Among these,
simple Bayes factor analogues of popular frequentist tests for accuracy of dynamic quantile forecasts have been developed.
To evaluate the relevant marginal likelihoods involved, either inappropriate assumptions have been made, or
pre-packaged multivariate adaptive quadrature methods have been employed, without accuracy assessment. This paper develops variational
Bayes methods to efficiently estimate lower bounds for the required marginal likelihoods. This facilitates a more accurate version of
one existing Bayesian test, and allows the development of a new test that out-performs its competing
Bayesian and frequentist analogues. The size and power properties of the proposed methods are examined via a simulation study,
illustrating favourable comparisons with existing testing methods. An empirical study employs
the proposed methods, in comparison with standard tests, to assess the adequacy of a range of
forecast models for Value at Risk (VaR) in several financial market data series.
\end{abstract} \noindent {\em Key words}:Bayesian Hypothesis
testing; Bayes factor; asymmetric-Laplace distribution; Value-at-Risk; quantile regression.

%\noindent {\em JEL Code:} Bayesian Hypothesis testing, MCMC, Value-at-Risk, quantile estimation.


%\newpage
%\setcounter{page}{2}

\section{Introduction}
%% Determining the accuracy of VaR models

Value-at-Risk (VaR) is now a standard measurement tool in risk management, used extensively by banking and other financial institutions.
It is an estimate (forecast) of the size of the minimum potential loss, over a given time horizon, with a specified probability,
for a financial position. VaR thus corresponds to the multiple of a quantile of the financial return distribution. It is a measure
that is widely used in practice for capital allocation, to protect against large negative market movements in asset prices,
following the Basel II Capital Accord. The accord further advises risk managers to regularly back-test their VaR forecast models,
using at least one year of historical data to compare actual returns with VaR forecasts. This paper proposes variational Bayes (VB) methods
to estimate the marginal likelihoods involved in Bayes factor methods, in the process developing more powerful back-testing methods.

Four well-known formal back-testing methods for quantile (VaR) forecasts are: the unconditional coverage (UC) test of Kupiec (1995); the
conditional coverage (CC) test of Christoffersen (1998); the dynamic quantile (DQ) test of Engle and Manganelli (2004); and the VaR Quantile
Regression (VQR) test of Gaglianone et al. (2011). Berkowitz, Christofferson and Pelletier (2011) developed a unified Lagrange
Multiplier framework for VaR assessment, incorporating these tests (except the VQR). Gaglianone et al. (2011) highlighted that the
VQR and DQ tests were over-sized in general, more-so in smaller samples, but have higher size-adjusted power than the UC and CC tests.
The latter outcome highlights that the VQR and DQ tests use more information, and in a mostly more effective manner, than the binary
variables that the UC and CC tests solely rely on; whilst the former result indicates a potential opportunity to develop better
and more accurate tests: this is the goal of this paper.

Bayesian methods for back-testing VaR forecasts are developed in Gerlach, Chen and Lin (2014), who develop a suite of
Bayesian methods that are roughly analogous to, and in simulations compare favourably with, the aforementioned four frequentist tests.
Their proposed methods are based on Bayes factors, requiring estimation of marginal likelihoods. The estimators proposed either
make an inappropriate assumption of Gaussianity ("for simplicity"), to allow analytic integration; or,
employ adaptive quadrature methods, which are not standard in the estimation of marginal likelihoods and can be inaccurate, especially in
highly concentrated likelihood situations. It is well known, see e.g. Berger (19xx) and Robert (20xx), that prior formulation plays a significant
role in Bayesian hypothesis testing and model selection. In this paper, the role of the prior choice is illustrated explicitly, whilst the
existing tests used and new tests developed are made somewhat robust to a wide range of possible prior variances, through an empirical size adjustment process.

The goal of this paper is to develop two further Bayesian testing methods for quantile forecast adequacy: one is a new test,
whilst the second attempts to improve on the accuracy of the quadrature-reliant method in Gerlach et al (2014). Both these tests
employ a method known as variational Bayes, to estimate lower bounds for the required marginal likelihoods. Variational Bayes (VB)
methods have become popular in the literature for fast and accurate approximation of intractable likelihoods in Bayesian settings,
e.g. see Ormerod and Wand (2010) and Girolami and Rogers (2006). As with the UC, CC, DQ and VQR tests, the proposed Bayesian tests
do not depend on the method of estimation of the model parameters involved in forecasting, and depend only weakly on the model that
generated the data.

The article is organized as follows: Section 2 reviews the evaluation of quantile forecast accuracy via existing frequentist
and Bayesian tests, as well as the variational Bayes method; Section 3 introduces the two new testing procedures;
Sections 4 and 5 present and discuss the results from a simulation study; Section 6 illustrates the results of an
empirical study comparing a range of competing VaR methods; concluding remarks appear in Section 7.

\section{Evaluating quantile forecast accuracy}
Let $y_t$ denote a financial asset return observation at time $t$, then VaR ($VaR_t$) at level $\alpha$ can be defined via:
\begin{eqnarray*}
\textrm{Pr}(y_t < \mbox{VaR}_t|\calF_{t-1}) = \alpha \, ,
\end{eqnarray*}
where $\calF_{t-1}$ is the information available at time $t-1$. For a forecast sample period, the
observed violation rate is the number of return observations that are more extreme than their
respective VaR forecast (i.e. $I_t = I(y_{t}< \mbox{VaR}_{t})=1$)), divided by the forecast sample size $T$.

There are many models and methods to estimate and forecast $\mbox{VaR}_{t}$ in financial return series. Regardless of which is employed,
the frequentist tests below can be applied to "back-test" each. This paper does not consider the UC or CC tests, as Galgianone et al (2011)
demonstrated their relatively low power compared to the DQ and VQR tests considered here.

\subsection{Frequentist testing}
Engle and Manganelli (2004) develop the DQ test, a joint test for correct rate and independence of violations. The null hypothesis
is: $ H_{0} \,:\, I_t$, $t=1,\ldots,T$ is an i.i.d. series with violation rate $\alpha$. A binary series of ``hits'',
$H_t = I_{t}-\alpha$, are then calculated. Under the null $E(H_t) = 0$ and $E(H_tW_{it})=0$, where $W_t$ contains an intercept plus
$q-1$ relevant explanatory variables, available at time $t-1$. The DQ test statistic examines whether all
parameters in a regression of $H$ on $W$ equal zero, calculated as:
\begin{eqnarray*}
\mbox{DQ}(q) = \frac{\utwi{H}'\utwi{W}\left(\utwi{W}'\utwi{W}\right)^{-1}\utwi{W}'\utwi{H}}{\alpha(1-\alpha)} \,\,.
\end{eqnarray*}
This statistic is analogous to a regression F statistic but including the intercept in the test. Under the null,
$DQ(q)$ tends to a $\chi^2_q$ distribution as $T$ gets large.
As in Engle and Manganelli (2004), we employ $p$ lagged hits and the VaR forecast at time $t$ as explanatory variables,
i.e. $W_t^{'} = (1, H_{t-1}, \mbox{VaR}_{t})$ ($p=1, q=3$, denoted as DQ1) and $W_t^{'} = (H_{t-1}, \ldots, H_{t-4}, \mbox{VaR}_{t})$
($p=4, q=6$, DQ4).

Gaglianone et al. (2011) employed direct ``Mincer-Zarnowitz'' quantile regression for the $\alpha$th conditional quantile of $y_t$:
\begin{eqnarray}\label{qreg}
Q_{y_t } (\alpha |\calF_{t-1}  ) = \beta_0  + \beta_1 VaR_t, \quad{\textrm{for all }}\alpha  \in {\rm{(0,1)}},
\end{eqnarray}
their test employs the actual data, not violation indicators, as well as the VaR forecast series itself, thus employing more information
than the UC, CC and DQ tests. If the VaR forecasts are accurate, then $\beta _0  = 0, \beta _1  = -1$ in \ref{qreg}.
Gaglianone et al. (2011) test the hypothesis: $\theta= (\beta_0, \beta_1-1)^{'} = \bf{0}$, using the statistic
$\mbox{VQR} = \hat{\theta}^{'} \left(\hat{\Sigma} \right)^{-1} \hat{\theta}$,
which asymptotically follows a $\chi^2_2$ distribution under the null hypothesis. We followed Gaglianone et al. (2011) and
Koenker and Machado (1999)'s recommendations here, in particular to estimate the matrix $\Sigma$. See those papers for details.

These are two of the most powerful, best performing tests in the literature to assess the accuracy of and back-test quantile forecasts.
In the next section two of the most powerful Bayesian tests in Gerlach et al (2014) are reviewed.

\subsection{Bayes Factor Testing}
In a Bayesian framework, hypothesis testing and model comparison problems can be tackled via marginal likelihoods, that are often
translated into Bayes factors (BFs). A general rule is that
$$ p(y|M_k) = \int p(y|\theta, M_k) p(\theta | M_k) d \theta \,\, ,$$
where model $M_k$ is generally preferred over $M_j$ if
$\mbox{BF} =\frac{p(y| M_k)}{p(y|M_j)} > 1$. BFs are also employed in hypothesis testing, i.e.  $H_{0}\, : \, \theta= \theta_0$,
where the hypothesis is conventionally rejected if $\frac{p(y|\theta_0, M)}{p(y|M)} < 1$. Berger (19x) and Robert (20xx) discuss how
the prior choice, well known to diminish with sample size Bayesian inference problems, can strongly affect Bayesian testing and model
selection. In the next section, this aspect is explicitly illustrated.

\subsubsection{A BF Dynamic Quantile test}
A Bayes factor requires a model and data distribution to produce a likelihood. The DQ test employs the series of
``hits'' $H_t = I_{t}-\alpha$, $t=1,\ldots,T$ and fits a regression: $ H_t = W_{t}\utwi{\beta} + \epsilon_t$, with $W$ defined previously and
containing $p=q-2$ lags of the hit series.
For a likelihood, a distribution needs to assumed for $\epsilon_t$. The simplest, but non-intuitive choice, as made by
Gerlach et al (2014), is $\epsilon_t \sim N(0,\sigma^2)$. This leads to:
$$
p(H|\beta,\sigma^2) = (2 \pi)^{-0.5(T-p)} (\sigma^2)^{-\frac{T-p}{2}} \exp\left( -\frac{1}{2\sigma^2} \sum_{t=p+1}^T \epsilon_t^2\right)
\,\, .
$$
A proper, conjugate Gaussian prior on $\beta$ is chosen, i.e. $\beta|\sigma^2 \sim N(0,C \sigma^2)$, where $C =cI$ is a diagonal matrix
with all diagonal elements $=c$, and a standard Jeffreys' prior is chosen for $\sigma^2$, i.e. $p(\sigma^2) \propto \sigma^{-2}$.
Choosing the prior for $\beta$ conditional on $\sigma^2$ allows this nuisance parameter to be
conveniently, analytically integrated out. The BF statistic is then:
\begin{eqnarray*}
\mbox{BFDQ} &=& \frac{p(H|\utwi{\beta}={\bf 0})}{p(H)} \\
            &=& \frac{\int p(H|\utwi{\beta}={\bf 0},\sigma^2) p(\sigma^2) d \sigma^2}{\int \ldots \int p(H | \utwi{\beta},\sigma^2)p(\utwi{\beta}|\sigma^2) p(\sigma^2) d \sigma^2 d \utwi{\beta}} \\
            &=& c^{0.5q} |A|^{0.5} (H'H-B'A^{-1}B)^{0.5(T-p)}(H'H)^{-0.5(T-p)},
\end{eqnarray*}
where $p = q-2$; $B = W'H$, $A=W'W+c^{-1}I$ and $|A|$ is the determinant of $A$.
Gerlach et al (2014) employ the same regressors as for the DQ statistics, giving BFDQ1 and BFDQ4 tests. Under the null,
all $\utwi{\beta} = {\bf 0}$, which is conventionally rejected whenever BFDQ$<1$. The integrals above can be analytically performed, first
because $p(H | \utwi{\beta},\sigma^2)p(\utwi{\beta}|\sigma^2)$ is proportional to a Gaussian distribution in $\utwi{\beta}$ and then
because both the resulting integrand and the numerator are proportional to an inverse gamma distribution in $\sigma^2$.
Smith and Kohn (1996) give an exposition of such Bayesian integrals in regression models.

The role of $c$ here is fairly explicit and clear: i.e. as $c \Rightarrow \infty$, $A \Rightarrow W'W$, $B'A^{-1}B \Rightarrow H'W(W'W)^{-1}W'H$
and $BFDQ \Rightarrow \infty$. Thus, as the prior variance $c \Rightarrow \infty$ the
null will never be rejected, known as Bartlett's paradox. Further, changing the value of $c$ will clearly shift the entire
distribution of BFDQ, and in a predictable manner if diag$(W'W)$ dominates $c^{-1}$, indicating that the conventional threshold of 1 may not
lead to an adequate or sensible decision rule here. To combat this, the use of empirical size adjusted thresholds, a common practice in
frequentist testing, is recommended, as discussed in a later section. Subsequently, we recommend $c=1000$ (so that diag$(W'W)$ will dominate
$c^{-1}$) is chosen, though we discuss some robustness properties of our results to different choices for $c$ in a later section.

Naturally, other choices are possible and more sensible here, other than Gaussian errors. In the next
section a new testing procedure that makes more reasonable assumptions is developed, that retains the same spirit as the DQ framework.

\subsubsection{A BF VQR test}
Koenker and Machado (1999) note that quantile regression estimation, usually performed by minimising the quantile distance function:
\begin{eqnarray*}
\stackrel{\rm min}{\utwi{\beta}} \,\, \sum_{t} u_{t} \left[\alpha-I(u_{t}<0)\right]
\end{eqnarray*}
is equivalent to a maximum likelihood (ML) estimation procedure when assuming i.i.d. skewed Laplace errors,
i.e. $u \sim SL(0,\sigma,\alpha)$, so that:
\begin{eqnarray*}
p_{\alpha}(u)=\frac{\alpha(1-\alpha)}{\sigma}\exp\left[-\left (\frac{u\left[\alpha-I(u<0)\right]}{\sigma}\right)\right].
\end{eqnarray*}
The ML and usual quantile regression estimates for $\utwi{\beta}$ are mathematically equivalent in this case.

The quantile regression model for the $\alpha$th conditional quantile of $y_t$ against its VaR forecast, can be written:
\begin{eqnarray*}
Q_{y_t} (\alpha |\calF_{t-1}  ) = \beta_0  + \beta_1 VaR_t, \quad{\textrm{for all }}\alpha  \in {\rm{(0,1)}} \,.
\end{eqnarray*}
To develop a Bayesian VQ procedure, again assuming a Jeffreys prior on $\sigma$ and integrating it out, Gerlach et al (2014) found:
\begin{eqnarray*}
p(\utwi{u}|\utwi{\beta}) = \alpha^T(1-\alpha)^T \Gamma(T) \left[\sum^{T}_{t=1} u_{t}(\alpha-I(u_t<0)) \right]^{-T} \,\, .
\end{eqnarray*}
Thus, the BFVQ statistic they develop is:
$$
\mbox{BFVQ} = \frac{ p({\bf u}|\beta_0=0, \beta_1=1)}{ \int \int p({\bf u}|\utwi{\beta}) p(\utwi{\beta}) d \beta_0 d \beta_1 }
$$
where the null of $\beta_0=0, \beta_1=1$ is rejected whenever $BFVQ<1$. The denominator above is a double integral over the
bivariate real line. Gerlach et al (2014) employ a highly diffuse, but proper, Gaussian prior on $\beta$, i.e. $N(0,cI)$.
The region of integration is then transformed to the region $(-1,1)\times(-1,1)$ and adaptive quadrature methods are employed to
numerically estimate this integral; they did not assess the accuracy of this method.

It is again possible to show that the term $c^{0.5q}$, where $q=2$ here, appears as a multiple
in the expression for BFVQ, which otherwise has no closed form due to the numerically approximated denominator. Thus the estimate of
BFVQ tends to $\infty$ as $c \Rightarrow \infty$ and the null will never be rejected in that case.
Again the distribution of BFVQ will be significantly shifted by changing $c$ and the threshold value of 1 is likely not an optimal decision rule.

This paper develops a new DQ style BF test and develops an alternative estimator for the marginal likelihood in the BFVQ method. Both these estimators employ the method of variational Bayes, now discussed.

\subsection{Variational Bayes}
\label{sec:VB}
Generally, the marginal likelihood $p(\cy|M)$ required in the BF is not analytically tractable. VB methods estimate a
lower bound for this quantity, optimised to be as close as possible to $p(\cy|M)$ under certain assumptions. In such cases, the
marginal log-likelihood, $\log p(\cy|M)$, can be expressed as:
$\mathcal{L}(q(\cz),\theta)+KL(q(\cz)||p(\cz|\cy))$ where, $\mathcal{L}(q(\cz),\theta)=\int\int q(\cz)\log\frac{p(\cz,\cy)}{q(\cz)}d\cz$ is
the expected joint log-likelihood of a set of relevant latent variables $\{\cz\}$ and $\cy$, with respect to a proposal
distribution $q(\cz)$; $\theta$ are the hyper-parameters.
Here, $KL$ is the Kullback-Leibler divergence between the proposal distribution and the posterior distribution of the latent variables.
Note that $\mathcal{L}(q(\cz),\theta)$ is a lower bound to the true log of the likelihood, since $KL>0$.
The Expectation-Maximisation (EM) algorithm proceeds by initially minimizing the KL divergence for a given set of hyper parameters,
$\theta$ (i.e. finding an appropriate $q(\cdot)$). Usually, this is done by setting $p(\cz|\cy,\theta)=q(\cz)$, in which
case $\log p(\cy|\theta)=\mathcal{L}(q(\cz),\theta)$.

However, an appropriate analytic distribution for the full posterior $p(\cz|\cy)$ often cannot be found. VB methods
(see e.g. Tzikas, Likas and Galatsanos, 2008 and Ormerod and Wand, 2010) differ from EM in that they focus far
more in maximising the lower bound $\mathcal{L}(q(\cz),\theta)$ and not on minimising the KL divergence. Additionally, to gain
tractability of $\mathcal{L}(q(\cz),\theta)$ it is often assumed that the proposal distribution can be factored such that:
$q(\cz_1,\cz_2;\theta)=q_1(\cz_1)q_2(\cz_2)\approx p(\cz_1,\cz_2|\cy,\theta)$.
It can be shown that for each proposal distribution
$\log q_i(\cz_i) \propto E_{q(\cz_{/i})}\clrbracket{\log p(\cy | \cz_i,\cz_{/i},\theta)+\log p(\cz_i)}$ where $\cz_{/i}$ is all latent
variables but the $i$-th, and the expectation $E$ is taken with respect to all $q(\cz_j)$ but the $i$-th. Due to the dependence
of the $i$-th proposal distribution on the rest of the proposal distributions, this algorithm is iterated until convergence.

It is clear that the effect of the prior variance $c$ on $\beta$ will shift the distribution of the BF statistics above and
that using the rule $BF<1$ may not be appropriate depending on $c$. The fact that VB methods estimate a lower bound for the
marginal likelihood $p(\cy|M)$ in the denominator of each BF means that using VB will result in systematically upwardly shifted BF estimates;
this would clearly also occur if $c$ was increased. In a subsequent section, the frequentist method of empirical size adjustment is employed
to mitigate the effects of both $c$ and $VB$ under-estimation, so as to choose more appropriate thresholds that lead to sensible and
(more) powerful (size-adjusted) BF methods.

\section{Two new Bayesian quantile testing procedures}
This section presents two new procedures that employ VB to estimate lower bounds for the required marginal likelihoods.

\subsection{A new Bayesian DQ procedure}
Engle and Manganelli (2004) employ a standard OLS estimation algorithm to a binary dependent variable regression. In this setting, it is
more typical to prefer a logistic or probit regression formulation. The standard probit specification allows a
convenient VB representation, as in Girolami and Rogers (2006), and MCMC framework in Albert and Chib (1993).
Keeping the DQ test notation above, a probit specification is:
$$ Pr\clrbracket{I_t=1 | \calF_{t-1} = \Phi \left(W_t^{'}\utwi{\beta}\right)},$$
with associated likelihood function:
$$
p({\bf I}|\utwi{\beta}) = \Pi_{t=p+1}^{n} \Phi\left(W_t^{'}\utwi{\beta} \right)^{I_t}
\left(1-\Phi\left(W_t^{'}\utwi{\beta} \right)\right)^{1-I_t} \,.
$$
Under the null of independent violations with rate $\alpha$, the intercept is
$\beta_0 = \Phi^{-1}(\alpha)$ and all slope parameters $\beta_i = 0$, $i=1,\ldots,p$. Thus,
the relevant BF here is:
\begin{eqnarray*}
\mbox{VBDQ} = \frac{ p({\bf I} | \beta_0 = \Phi^{-1}(\alpha), \beta_1=0,\ldots,\beta_p)}
              { \int \ldots \int p({\bf I}|\utwi{\beta}) p(\utwi{\beta}) d \utwi{\beta} } \,\, ,
\end{eqnarray*}
where the null is conventionally rejected whenever $VBDQ<1$.

The denominator is a multiple integral over the real line, in $q$ dimensions. As in Gerlach et al (2014), a highly diffuse, but proper,
Gaussian prior on $\utwi{\beta}$ is chosen, i.e. $\utwi{\beta} \sim N(0,cI)$, where $c$ is finite; the
integral is still intractable under that choice. Employing the usual Gaussian latent variable representation of the probit specification,
the VB lower bound estimate of Girolami and Rogers (2006) is employed, as derived in \secref{class_beta} and \secref{class_y},
yielding the lower bound for the denominator in VBDQ as:
\begin{align}
\mbox{LBDQ} = \exp \left[-\chalf{}\clrbracket{T\log 2\pi+q\log c+\frac{\cmu'\cmu+Tr(\cSigma)}{c}+
\sum_{t=1}^{T}\cW_t'\cSigma\cW_t-q-\log|\cSigma|-\sum_{t=1}^{T}\log C_t} \right] \,\, ,
\end{align}
where $T, W_t$ are defined as for the BFDQ statistics, while $\cmu, \cSigma$ are derived in \secref{class_beta} - \secref{class_lik},
and are: $\cSigma= (\cW'\cW+\frac{1}{c}\cI)^{-1}$, $\cmu =\cSigma\cW'\clrangle{\cy^*}$, where $\cy^*$ is the Gaussian latent variable
and the notation $\clrangle{\cdot}$ is employed to denote an expectation under a relevant VB distribution $q(\cdot)$.
See \secref{class_beta} - \secref{class_lik} for details. Choosing $p=1$ lagged hits gives the VBDQ1, $p=4$ lagged hits in $W$ gives the
VBDQ4 statistic.

The lower bound for the denominator will clearly tend to 0 as $c \Rightarrow \infty$, due to the term $\exp(-0.5q\log c) = c^{-0.5q}$.
Thus, $c^{0.5q}$ again appears as a multiple in the VBDQ BF, thus the estimate of VBDQ tends to $\infty$ as $c \Rightarrow \infty$
and the null will never be rejected in that case.

As a reference, adaptive quadrature can also be employed to estimate the integral in the denominator of VBDQ, as for the BFVQ method; this
is done with $p=1 (q=3)$ only and the resulting BF is denoted AQDQ1. The accuracy of the VBDQ1 method is also assessed by considering the
marginal likelihood estimator in Chib (1995), based on the MCMC sampler for the probit regression model in Albert and Chib (1993). Kass and
Raftery (1995) proposed Laplacian approximation to approximate marginal likelihoods. This
involves a 2nd order Taylor series expansion of the integrand $p({\bf I}|\utwi{\beta}) p(\utwi{\beta})$, about
its mode, here found through a standard numerical search algorithm. This approximation to the integrand is proportional to a
Gaussian distribution in $\utwi{\beta}$, allowing the approximated integral to be done analytically. We follow this approach, with $p=1$,
denoting the resulting statistic KRDQ1.

\subsection{VB approximation to BFVQ statistic}
\label{sec:vb_derivation}
The BFVQ statistic in Gerlach et al (2014) (and above) is:
$$
\mbox{BFVQ} = \frac{ p({\bf u}|\beta_0=0, \beta_1=1)}{ \int \int p({\bf u}|\beta) p(\beta) d \beta_0 d \beta_1 }
$$
The denominator, where $u$ follows an Asymmetric Laplace distribution, is an intractable integral. We propose to approximate a lower bound for this
integral by the method of VB, which is a novel contribution to the literature.

We begin by restating the Asymmetric Laplace (AL) distribution as a continuous mixture of Gaussian distributions.
The Bayesian hierarchy is as follows:
\begin{align}
\nonumber y_t|\cbeta,w_t,\sigma,\alpha & \sim \cN\clrbracket{\cW_t'\cbeta+\frac{1-2\alpha}{\alpha(1-\alpha)}\sigma w_t,
\frac{2}{\alpha(1-\alpha)}\sigma^2 w_t}\\
\nonumber \cbeta &\sim \cN(0,c \cI)\\
\nonumber p(w_t)&=\exp(-w_t)\\
p(\sigma)&\propto\frac{1}{\sigma}
\label{eq:BFVQ_hier}
\end{align}

For brevity, subsequently the notation $\clrangle{\cdot}$ is employed to denote an expectation under $q(\cdot)$.
The approximate distribution $q(\cbeta)$ is $\cN(\cmu,\cSigma)$ where \footnote{Derivation is shown in \secref{q_beta}},
\begin{align}
\cSigma &=\clrbracket{\frac{\caexp}{2}\clrangle{\frac{1}{\sigma^2}}\csum\clrangle{\frac{1}{w_t}}\cW_t'\cW_t+\frac{1}{c}\cI}^{-1}\\
\cmu&=\cSigma\clrbracket{\csum\clrbracket{\frac{\caexp}{2}\clrangle{\frac{1}{\sigma^2}}\clrangle{\frac{1}{w_t}}y_t-\frac{1-2\alpha}{2}\clrangle{\frac{1}{\sigma}}}\cW_t}
\end{align}

The approximate distribution $q(\cw)$ can be factorised as independent components such that $q(\cw)=\prod_{t=1}^{T}q(w_t)$, where
$q(w_t) = \cGIG\clrbracket{\chalf{},\alpha,\beta}$ where,
\begin{align}
\alpha_t & =\clrbracket{\frac{(1-2\alpha)^2}{2\caexp}+2} \\
\beta_t & = \frac{\caexp}{2}\clrangle{\frac{1}{\sigma^2}}\clrbracket{y_t^2-2y_t\cW_t'\clrangle{\cbeta}+\cW_t'\clrangle{\cbeta\cbeta'}\cW_t}
\end{align}

The exact VB distribution over $q(\sigma)$, though tractable, is susceptible to numerical problems due to the dependence of the
normalising constant on the parabolic cylindrical function. Hence, we instead limit $q(\sigma)$ to an Inverse Gamma
distribution, $IG(a,b)$. In order to obtain the parameters $a,b$ we are required to maximise the following function
using the given derivatives \footnote{Full derivation is shown in \secref{q_sigma}}.

\begin{align}
F_\sigma&=(a-T)(\log b-\psi(a))+(b-\gamma)\frac{a}{b}-\delta \frac{a(a+1)}{b^2}-a\log b+\log \Gamma(a)\\
\frac{\partial F}{\partial a}&=(T-a)\psi^{(1)}(a)-\frac{\gamma}{b}-\frac{\delta(2a+1)}{b^2}+1\\
\frac{\partial F}{\partial b}&=-\frac{T}{b}+\frac{\gamma a}{b^2}+\frac{2\delta a(a+1)}{b^3}
\end{align}

The expected log-likelihood of the set of latent variables $\mathcal{L}$ is a lower bound for the
marginal likelihood. Thus, using the approximate distributions derived, it can be shown that
\begin{align}
\nonumber \mbox{LBVQ} = &-\chalf{}\Biggl[T\log\frac{2}{\caexp} +2T(\log b -\psi(a))+q \log (c)-\cmu'\cSigma^{-1}\cmu\\
\nonumber&+\frac{\caexp}{2}\frac{a(a+1)}{b^2}\sum_{t=1}^{T}\sqrt{\frac{\alpha_t}{\beta_t}}y_t^2+\clrbracket{\frac{(1-2\alpha)^2}{2\caexp}+2}\csum\clrbracket{\sqrt{\frac{\beta_t}{\alpha_t}}+\frac{1}{\alpha_t}}\\
&-(1-2\alpha)\frac{a}{b}\csum y_t\Biggr]
+\chalf{}\clrbracket{\log|\cSigma|}-\chalf{}\csum\log\alpha_t+\frac{T}{2}+\log \Gamma(a)-a\psi(a)+a
\end{align}
is a lower bound \footnote{Full derivation can be seen in \secref{l_bound}}. Exponentiating this bound, and
employing it as the denominator in BFVQ above yields the statistic denoted VBVQ.

LBVQ, the lower bound for the log-likelihood, will tend to $-\infty$, so the denominator in VBVQ will clearly tend to 0,
as $c$ tends to $\infty$, since the term $c^{-0.5q}$ is again present; thus the estimate of VBDQ tends to $\infty$ in that case and
the null will never be rejected.

%$q(\cbeta,\cw, \sigma)=q(\cbeta)q(\cw)q(\sigma)\approx p(\cbeta,\cw, \sigma|\cy)$ where $\cw$ is an extra set of latent variables
%which will be introduced in \secref{vb_derivation}. As the KL divergence is always positive, $\mathcal{L}(q(\cbeta,\cw, \sigma),\theta)$
%forms a lower bound. In the M-step this lower bound $\int\int q(\cbeta,\cw, \sigma|\theta^{OLD})\log p(\cbeta,\cw, \sigma,\cy)d\cbeta dz$
%is maximised w.r.t. the hyper-parameters. The E and M steps are iterated over until convergence. However, since all possible parameters
%are marginalised over here, there will be no M-step.
%
%There exists a closed form solution for $q(\cz_i)$ under VB where $\cz_i$ is the i-th member of the set of latent variables $\cz$.
%Thus we let $q(\cz_i)=\frac{1}{Z}\exp(E_{q(\cz_{\backslash i})}[p(\cz,\cy)])$ where $\cz_{\backslash i}$ is the set of all
%latent variables except for the i-th member and $Z$ is the normalising constant. Thus under VB, the approximate distribution
%is the expectation of the joint probability between all latent variables and the observed values, w.r.t. the other
%approximate distributions. Since, an approximate distribution on one latent variable depends on the other
%approximate distributions, this step is iterated until convergence.

To assess the accuracy of the VBVQ method, direct prior simulation can be employed to estimate the marginal likelihood: i.i.d. Monte Carlo
iterates are simulated from $\cbeta \sim \cN(0,c \cI)$, then employed to produce iterates of the likelihood, $p({\bf u}|\utwi{\beta})$.
These are simply averaged to produce an estimate of the marginal likelihood. Since the integral is only in two dimensions, this method
seems to work reasonably consistently in large Monte Carlo samples. The method of Gelfand and Dey (1990) is also employed as
a reference. This method takes a random walk Metropolis sample from the posterior $p(\utwi{\beta} | {\bf u})$, whose
iterates are employed to produce iterates of the likelihood $p({\bf u}|\utwi{\beta})$. The average of the inverse of these likelihood iterates, 
is an estimate of the inverse of the marginal likelihood.

Exactly as for the DQ test, the BFDQ and VBDQ statistics above cannot be conducted when the number of violations equals 0.
However, this restriction does not affect the VQ, BFVQ or VBVQ tests.

\section{Simulation study}
The empirical properties of the proposed Bayesian methods are assessed, and compared to existing tests, via a simulation study.
The simulation setting as in Gaglianone et al. (2011) and Gerlach et al (2014) is employed. The true model is a GARCH(1,1), specified as:
\begin{eqnarray*}
\sigma_t^2 = 0.1 + 0.1y_{t-1}^2 + 0.85\sigma_{t-1}^2 \,\,;\,\, y_t = \sigma_t \epsilon_t \,;\, \epsilon_t \sim N(0,1) \\
\end{eqnarray*}
where $\mbox{VaR}_{t,\alpha} = \sigma_t \Phi^{-1}(\alpha)$. To assess power an incorrect, but commonly applied, historical simulation (HS)
VaR estimator is employed in the alternative hypothesis:
\begin{eqnarray*}
\mbox{HS}250_{t,\alpha} = \hat{Q}_{\alpha}(y_{t-250},\ldots,y_{t-1})
\end{eqnarray*}
using the sample percentile of the last 250 observations as a 1-step-ahead VaR forecast. 25000 replications of data, using a burn-in period of
$n=250$ observations, followed by forecast sample sizes of $T=250, 500, 1000$ and $2500$ observations, are simulated from this model.
For each data set, the DQ1, DQ4 and VQR tests are conducted. Further, the BFDQ1, BFDQ4 and BFVQ statistics from Gerlach et al (2014) and
the proposed VBDQ1, VBDQ4, AQDQ1, KRDQ1 and VBVQ statistics are also calculated for each data set. The choice $c=1000$ is made for all
the BF statistics. The statistics are all calculated first under the null, using the true $\mbox{VaR}_{t,\alpha}$ series,
for $t=n+1,\ldots,n+T$, then calculated under the alternative, using the estimated $\mbox{HS}250_{t,\alpha}$ series.
$\alpha=0.05, 0.01$ are used for the quantile levels, though most interest is in $\alpha=0.01$.

As in Gerlach et al (2014), to compare the methods on an equal footing, size and power as well as empirically adjusted
size and size-adjusted power are considered. This is standard practice when comparing frequentist tests, but is not standard for Bayesian
methods; however this will allow direct, objective comparison of all methods considered on an equitable basis. It will also
allow us to estimate sensible thresholds for the BF statistics, that account for the choice of prior variance $c$ and also for
the upward bias in BF estimators employing a VB lower bound. For a truly continuous statistic, the adjusted size can always be made
exactly equal to nominal (here 0.05); this applies to the VQ, BFVQ and VBVQ statistics. However, the DQ-type statistics are subject
to some discreteness, being based on regressing a binary "hit" series that may have only a few (e.g. 1 or 2) "success" observations on
mostly binary regressors, also with few "successes"; i.e. there may be only 1 or 2 observed violations, especially when $\alpha=0.01$ and
$T=250, 500$. In those cases, an exact size adjustment may not be possible. Thus, adjusted size is reported below.

%Without loss of generality, for each of the five BF statistics, we subsequently consider $BF*=BF^{-1}$, and reject the null when $BF*>1$.
%This simply allowed comparison of the BF statistics with the frequentist tests, for all of which higher values lead to rejection.
%Thus, all estimated threshold values presented are based on $BF*$; simply taking their reciprocal gives the thresholds for the
%original BF statistics.
Table \ref{size5} shows the empirical estimates for size, as well as adjusted size, across the methods employed at
$\alpha=0.05$ for $T=250,500$. Also shown are the empirical $5\%$ points ("threshold") for all methods, calculated via
sample percentiles across the 25000 replications for each statistic. These are the thresholds used to calculate the
empirically adjusted size and size-adjusted power below; i.e. adjusted size is the observed percentage
of test statistics, across the 25000 replications, that are beyond the empirical $5\%$ threshold calculated under the null hypothesis.
Size-adjusted power is the same observed percentage, using the same empirical threshold, employing the test statistics calculated under the
incorrect HS VaR estimator. There is no reason why the value 1 should be the 95th percentage point for the sampling distribution of the
employed BF statistics, so size for the BF methods is not particularly relevant, but is reported as a reference for comparison.
Adjusted size is, however, relevant to the comparison of all methods. Since the DQ type tests cannot be run when there are 0 violations,
all results for methods with DQ in their title below are only for datasets when the violation rate $>0$. When $T=250$ there were 2126
datasets with 0 violations; for $T = 500$ there were 157 datasets with 0 violations; and for $T \ge 1000$ no 0 violation datasets 
were generated. The results for methods with VQ in the title are for all 25000 datasets in each case.

\begin{table}[thp]
\begin{center}
\caption{Size, empirical threshold and adjusted size for nominal 5\% tests of 1-step-ahead quantile forecasts at $\alpha=0.05$.
} \label{size5}
\begin{tabular}{lcccccc}
\hline
 & \multicolumn{3}{c}{T=250} & \multicolumn{3}{c}{T=500}                    \\
Method  &        Size  &           threshold &   adj. size &         Size &           threshold &  adj. size\\ \hline
DQ1     &\cblue{0.0538}&               8.018 &\cblue{0.0499}&\cblue{0.0472}&               7.625 & \fbox{0.0500}   \\ [1.3pt]
BFDQ1   &       0.0007 &  1.247$\times10^{5}$& \fbox{0.0500}&       0.0000 &  5.419$\times10^{5}$&\cblue{0.0499}   \\ [1.3pt]
VBDQ1   &       0.0000 &  1.846$\times10^{5}$& \fbox{0.0500}&       0.0000 &  6.462$\times10^{5}$& \fbox{0.0500}   \\ [1.3pt]
KRDQ1   &       0.0010 &              153.73 & \fbox{0.0500}&       0.0002 &             4723.89 & \fbox{0.0500}   \\ [1.3pt]
AQDQ1   &       0.00004&             1688.67 &\cblue{0.0499}&       0.0000 &              227.42 & \fbox{0.0500}   \\ [1.3pt]
DQ4     &       0.0647 &              13.777 & \fbox{0.0500}&\cblue{0.0532}&              12.816 & \fbox{0.0500}   \\ [1.3pt]
BFDQ4   &       0.0005 &  7.581$\times10^{9}$& \fbox{0.0500}&       0.0000 & 1.340$\times10^{11}$& \fbox{0.0500}  \\ [1.3pt]
VBDQ4   &       0.0000 & 2.820$\times10^{10}$& \fbox{0.0500}&       0.0000 & 1.879$\times10^{11}$& \fbox{0.0500}   \\ [1.3pt]
VQR     & \fbox{0.0488}&               5.923 & \fbox{0.0500}&\cblue{0.0556}&               6.291 &\cblue{0.0499}   \\ [1.3pt]
BFVQ    &       0.0654 &               0.362 & \fbox{0.0500}& \fbox{0.0520}&              25.751 & \fbox{0.0500}   \\ [1.3pt]
VBVQ    &       0.0428 &               2.033 & \fbox{0.0500}&       0.0363 &               3.772 & \fbox{0.0500}   \\ [1.3pt]
\hline
\end{tabular}
\par\smallskip
\parbox{.9\textwidth}{ The correct, nominal thresholds are: VQR 5.99; DQ1 7.81; DQ4 12.59}
\end{center}
\end{table}

At $T=250$, only the DQ1 and VQR tests achieve close to nominal size, with $5.4\%, 4.9\%$ respectively and empirical thresholds close to
their nominal values, whilst the DQ4 is quite over-sized and most BF methods are well under-sized (from a frequentist point of view);
however all methods achieve correct adjusted sizes of (almost or) exactly $5\%$. For $T=500$ the BFVQ, DQ1 and DQ4 tests have the closest to
nominal size ($5.2\%, 4.7\%, 5.3\%$), followed by VQR; again all methods achieve the corrected adjusted size exactly equal to nominal.

Table \ref{size51} shows the empirical estimates for size and then adjusted size across all the methods employed at $\alpha=0.05$
for $T=1000,2500$. Also shown are the empirical $5\%$ points for the methods across the 25000
replications used to calculate the adjusted size and size-adjusted power.

\begin{table}[thp]
\begin{center}
\caption{Size, empirical threshold and adjusted size for nominal 5\% tests of 1-step-ahead quantile forecasts at $\alpha=0.05$.
} \label{size51}
\begin{tabular}{lcccccc}
\hline
 & \multicolumn{3}{c}{T=1000} & \multicolumn{3}{c}{T=2500}                    \\
Method   &        Size  &           threshold &   adj. size &        Size  &          threshold  &    adj. size   \\ \hline
DQ1      &\cblue{0.0448}&               7.566 & \fbox{0.0500}&\cblue{0.0435}&              7.541  & \fbox{0.0500}   \\ [1.3pt]
BFDQ1    &       0.0000 &  1.247$\times10^{5}$& \fbox{0.0500}&       0.0000 & 8.606$\times10^{6}$ &\cblue{0.0499}   \\ [1.3pt]
VBDQ1    &       0.0000 &  1.846$\times10^{5}$& \fbox{0.0500}&       0.0000 & 8.439$\times10^{6}$ &\cblue{0.0499}   \\ [1.3pt]
KRDQ1    &       0.0000 &              153.21 & \fbox{0.0500}&       0.0000 & 1.074$\times10^{6}$ &\cblue{0.0499}   \\ [1.3pt]
AQDQ1    &       0.0000 &             1688.67 & \fbox{0.0500}&       0.0000 & 1.033$\times10^{7}$ & \fbox{0.0500}  \\ [1.3pt]
DQ4      & \fbox{0.0494}&              12.552 & \fbox{0.0500}&\cblue{0.0448}&             12.345  & \fbox{0.0500}   \\ [1.3pt]
BFDQ4    &       0.0000 &  7.581$\times10^{9}$& \fbox{0.0500}&       0.0000 & 3.055$\times10^{13}$& \fbox{0.0500}  \\ [1.3pt]
VBDQ4    &       0.0000 & 2.820$\times10^{10}$& \fbox{0.0500}&       0.0000 & 2.528$\times10^{13}$& \fbox{0.0500}  \\ [1.3pt]
VQR      &\cblue{0.0540}&               6.213 & \fbox{0.0500}& \fbox{0.0453}&               5.788 & \fbox{0.0500}  \\ [1.3pt]
BFVQ     &       0.0450 &               0.362 &\cblue{0.0499}&       0.0210 &              59.075 & \fbox{0.0500}  \\ [1.3pt]
VBVQ     &       0.0369 &               2.033 & \fbox{0.0500}&       0.0313 &               9.942 & \fbox{0.0500}   \\ [1.3pt]
\hline
\end{tabular}
\par\smallskip
\parbox{.9\textwidth}{ The correct, nominal thresholds are UC, IND 3.84; CC, VQ 5.99; DQ1 7.81; DQ4 12.59}
\end{center}
\end{table}

At $T=1000$, DQ4 is closest to nominal size with $4.9\%$, while VQR, BFVQ and DQ1 are also reasonably close to nominal size; all methods
achieve correct adjusted sizes exactly equal to $5\%$. For $T=2500$ the VQR, DQ4 and DQ1 tests have closest to nominal sizes
($4.53, 4.35\%$ respectively) and for all methods the corrected size is almost exactly the nominal $5\%$ level.

\begin{figure}[thp]
     \centering
      \includegraphics[width=1.0\textwidth]{DQvsBFDQ1vsVBDQ_final.png}
\caption{\label{powerDQBF250} Minus the logarithm of the VBDQ1 (dark grey diamond) and BFDQ1 (light grey asterisk) and logarithm of the
DQ1 (black square) statistics, plot against number of violations under the HS quantile estimator, $T=250$; (a) $\alpha =5\%$; 
(b) $\alpha =1\%$. The horizontal lines are minus the logarithm of the empirical 5\% points of the VBDQ1 (dark grey), BFDQ1 (light grey), 
and logarithm of the 95\% threshold of the DQ1, statistics.}
\end{figure}

Tables \ref{power51} - \ref{power52} show the empirical estimates for power and size-adjusted power (SAP) across the methods
employed at $\alpha=0.05$. At $T=250$ Gerlach et al (2014) found that the BFDQ1 was marginally more powerful, size-adjusted, than
the DQ1 and DQ4 tests; the result is replicated here. However, the proposed VBDQ1 statistic outperforms all tests with the highest
SAP here. To examine this result in more detail, consider Figure \ref{powerDQBF250}(a), showing minus the logarithm of
the VBDQ1 and BFDQ1 statistics and the logarithm of the DQ1 statistic, plot against the number of violations under the alternative hypothesis;
i.e. under the HS model. All three methods show U-shaped behavior in the locations of the distributions against violation numbers. 
As expected, all three methods reject the null for very low and very high numbers of violations; i.e. compared to the mean of 12.5 
violations at $\alpha=0.05$, $T=250$ under the null. However, all three tests also have plenty of rejections for violation numbers 
close to expected (e.g. (9,15)): likely rejecting violation series showing "significant" serial dependence. 
Gerlach et al (2014) found that the extra power of BFDQ1 over DQ1 is achieved through rejecting many more series when the number of 
violations is comparatively low, specifically for 3-15 violations; the DQ1 has more rejections than BFDQ1 when 16-27 violations are 
observed, but the differences in rejection frequencies for these higher violations are much smaller than
they are for 3-15 violations; as illustrated in Figure \ref{pdqBF250}(a), showing the rates of rejection at each number of violations for the
three methods. Here the superior size-adjusted power of the VBDQ1 is well illustrated: from 1-15 violations the VBDQ1 mimics, or exceeds, the
higher rejection rates of the BFDQ1 test, compared to the DQ1 test; whilst from 16-31 violations the VBDQ1 mimics, or is just below,
the higher rejection rates achieved by the DQ1 test, compared to the BFDQ1 test. For violation numbers of 1,2 or those higher than 30 
all three methods almost always reject the null. Thus, at $T=250$ and $5\%$ VaR forecasting, the VBDQ1 and BFDQ1 tests
appear to have much higher power than the DQ1 at detecting dependence in the violation series when the number of 
violations is relatively small ($\sim 3-15$ violations for VBDQ1), however the VBDQ1 and DQ1 statistics have similarly higher 
size-adjusted power than the BFDQ1 when the number of violations is relatively large (16-31 violations); thus contributing to an 
overall much higher size adjusted power in Table \ref{power5} when $T=250$ for the VBDQ1 method.

The VBDQ1 method also has much higher SAP than the KRDQ1, AQDQ1, VBVQ, BFVQ and VQR tests at $T=250$.
The values of SE (most $\approx 0.003$), thanks to 25000 replications, roughly suggest that only very, very small pairwise differences in 
SAP will be significant at a $5\%$ level. At the suggestion of a referee, we run all pairwise likelihood ratio (LR) tests for 
equality of size-adjusted power. The details of the test are in an appendix. There are 55 unique pairs among the 11 methods. 
None of the 55 associated tests for equal SAP has a p-value $> 0.05$, in fact 54 of the p-values are $\approx 0$, suggesting 
clear statistically significantly differences. The highest p-value is for the difference between DQ1 and DQ4's SAP 
(p-val = 0.018, difference = 0.0085). In particular, the top three ranked methods, VBDQ1, VBDQ4 and BFDQ1 are all statistically different 
to each other, and to all eight other methods, in terms of size-adjusted power with p-values $= 0$ to the precision of Matlab on my PC.

\begin{table}%[thp]
\begin{center}
\caption{Power and size-adjusted power for nominal 5\% tests of 1-step-ahead quantile forecasts at $\alpha=0.05$ and $T=250, 500$.
}\label{power51}
\begin{tabular}{lcccccc}
\hline
 & \multicolumn{3}{c}{T=250} & \multicolumn{3}{c}{T=500}                   \\
Method   &      Power  &   Size-adj. &   SE   &       Power &   Size-adj. &    SE         \\ \hline
DQ1      &       0.358 &       0.346 & 0.0030 &\cblue{0.449}&       0.462 & 0.0032 \\ [1.3pt]
BFDQ1    &       0.019 &\cblue{0.420}& 0.0031 &       0.007 &\cblue{0.505}& 0.0032 \\ [1.3pt]
VBDQ1    &       0.002 & \fbox{0.478}& 0.0032 &       0.001 & \fbox{0.533}& 0.0032 \\ [1.3pt]
KRDQ1    &       0.018 &       0.203 & 0.0025 &       0.013 &       0.170 & 0.0024 \\ [1.3pt]
AQDQ1    &       0.003 &       0.167 & 0.0024 &       0.003 &       0.179 & 0.0024 \\ [1.3pt]
DQ4      &       0.381 &       0.337 & 0.0030 & \fbox{0.513}&\cblue{0.503}& 0.0032 \\ [1.3pt]
BFDQ4    &       0.008 &       0.365 & 0.0030 &       0.002 &\cblue{0.515}& 0.0032 \\ [1.3pt]
VBDQ4    &       0.000 &\cblue{0.448}& 0.0031 &       0.000 &\cblue{0.495}& 0.0032 \\ [1.3pt]
VQR      &       0.159 &       0.161 & 0.0023 &       0.275 &       0.265 & 0.0028 \\ [1.3pt]
BFVQ     &       0.328 &       0.289 & 0.0029 &\cblue{0.367}&       0.356 & 0.0030 \\ [1.3pt]
VBVQ     &       0.362 &       0.387 & 0.0031 &\cblue{0.373}&       0.423 & 0.0031 \\ [1.3pt]
\hline
\end{tabular}
\par\smallskip
\parbox{.9\textwidth}{ SE is $\sqrt{\frac{p(1-p)}{25000}}$, where $p$ is size-adjusted power.}
\end{center}
\end{table}

\begin{table}%[thp]
\begin{center}
\caption{Power and size-adjusted power for nominal 5\% tests of 1-step-ahead quantile forecasts at $\alpha=0.05$ and $T=1000, 2500$.
}\label{power52}
\begin{tabular}{lcccccc}
\hline
 & \multicolumn{3}{c}{T=1000} & \multicolumn{3}{c}{T=2500}                   \\
Method   &       Power &  Size-adj.  &   SE   &       Power &    Size-adj. &   SE    \\ \hline
DQ1      &\cblue{0.628}&\cblue{0.644}& 0.0030 &\cblue{0.944}&\cblue{0.950} & 0.0014 \\ [1.3pt]
BFDQ1    &       0.010 &\cblue{0.681}& 0.0030 &       0.045 &\cblue{0.956} & 0.0013 \\ [1.3pt]
VBDQ1    &       0.002 &\cblue{0.651}& 0.0030 &       0.025 &\cblue{0.957} & 0.0013 \\ [1.3pt]
KRDQ1    &       0.007 &       0.265 & 0.0030 &       0.032 &       0.899  & 0.0019 \\ [1.3pt]
AQDQ1    &       0.005 &       0.355 & 0.0030 &       0.036 &       0.864  & 0.0022 \\ [1.3pt]
DQ4      & \fbox{0.705}&\cblue{0.706}& 0.0029 &\fbox{0.966}& \fbox{0.969} & 0.0011 \\ [1.3pt]
BFDQ4    &       0.003 & \fbox{0.713}& 0.0029 &       0.017 & \fbox{0.969} & 0.0011 \\ [1.3pt]
VBDQ4    &       0.000 &\cblue{0.629}& 0.0030 &       0.003 &\cblue{0.960} & 0.0012 \\ [1.3pt]
VQR      &       0.484 &       0.468 & 0.0032 &       0.875 &       0.886  & 0.0020 \\ [1.3pt]
BFVQ     &\cblue{0.421}&       0.442 & 0.0031 &       0.768 &       0.902  & 0.0019 \\ [1.3pt]
VBVQ     &       0.522 &       0.578 & 0.0031 &       0.883 &\cblue{0.926} & 0.0017 \\ [1.3pt]
\hline
\end{tabular}
\par\smallskip
\parbox{.9\textwidth}{ SE is $\sqrt{\frac{p(1-p)}{25000}}$, where $p$ is size-adjusted power.}
\end{center}
\end{table}

At $T=500$, the five methods with highest size-adjusted power for 5\% quantile forecasting are: VBDQ1, BFDQ4, BFDQ1, DQ4 and the VBDQ4 test
with $\sim 50-53\%$. The VBDQ1 is only marginally preferred here and the top five tests are very close in size-adjusted power (SAP).
The BFDQ4 (BFDQ1) test rejects more often than the DQ4 (DQ1) test when the number
of violations is between 8 and 27 (12 and 28; 25 violations are expected at $\alpha=0.05, T=500$), whilst the DQ4 rejects more for
28-46 (DQ1 for 29-42) violations: again the BFDQ statistics have slightly more power to detect dependent violations at lower numbers of
violations, compared to the DQ tests; however, in this case things even out so that the BFDQ and DQ method's size-adjusted powers
are close to comparable in each case. The VBDQ1 method has rejection rates similar to those of the BFDQ1 for lower violations and 
closer to that of the DQ1 for higher violations, allowing it to have marginally the highest SAP overall. The 55 pairwise tests for equality
of size-adjusted power yield 54 p-values $\approx 0$ (highest is $0.001$), with only the difference between BFDQ1 and DQ4 giving an
insignificant p-value of $0.49$ (observed difference $0.002$). The VBDQ1 has SAP that is significantly different to all other tests.

For $T=1000$, the DQ and VQR tests have close to nominal size, so their SAPs and power are very similar. Overall, the BFDQ4
has marginally the highest SAP ($\approx 71\%$), followed closely by the DQ4 and BFDQ1, then
VBDQ1 and DQ1 methods. The pairwise tests for equality of size-adjusted power yield all 55 p-values $\approx 0$, highest 
p-value $6.86 \times 10^{-5}$. Figure \ref{pdq1000} shows rejection rates for the VBDQ1, BFDQ1, BFDQ4, DQ1 and DQ4 methods.
The BFDQ4 (BFDQ1) test rejects more than the DQ4 (DQ1) test when the number of violations is between
33 and 52 (33 and 55; 50 violations are expected), whilst the DQ4 (DQ1) rejects more than the BFDQ4 (BFDQ1) for 53-72 (DQ1 for 56-71)
violations: again the BFDQ statistics have slightly more power to detect correlation in violations at lower numbers of violations, 
compared to the DQ tests, with the result reversed for higher violation numbers. The VBDQ1 method has rejection rates marginally lower than
those of the BFDQ1 for most violations (36 to 63) and closer to that of the DQ1 for a small number of higher violations (64-69),
meaning it overall has marginally lower size-adjusted power than the BFDQ and DQ tests here. Figure \ref{pvq1000} shows rejection
rates for the VBVQ, BFVQ and VQR methods, plot (a) is for $5\%$. Clearly, at almost all violation numbers the VBVQ has a
marginally higher rejection rate than BFVQ and VQR, whilst BFVQ is superior to VQR up until 50 violations, but inferior at higher 
violation numbers, 51-65, then BFVQ again outperforms VQR for higher violations.

For $T=2500$, similar results to $T=1000$ are obtained, but with a marked increase in SAP for all the
methods. Now the BFDQ4 and DQ4 methods have equal highest SAP, $\approx 97\%$, followed closely by VBDQ4, VBDQ1, BFDQ1 and DQ1 tests, all with
SAP $\in 0.95-0.96$. The pairwise tests for equality of size-adjusted power yield 52 p-values $<0.05$, with 50 $\approx 0$. The
difference between the top ranked BFDQ4 and DQ4 has a p-value $=1$ to the precision of Matlab. The other insignificant differences are for
the pairs BFVQ and AQDQ1 (p-val $0.227$, observed difference $0.003$) and BFDQ1 and VBDQ1 (p-val $0.103$, observed difference $0.001$).

\begin{figure}%[htp]
     \centering
      \includegraphics[width=0.8\textwidth]{pdqBFVB250_final.png}
\caption{\label{pdqBF250} Rates of rejection for the VBDQ1 (dark grey diamond), BFDQ1 (light grey asterisk) and DQ1 (black square) statistics,  
for each number of violations under the HS quantile estimator, $T=250$; (a) $\alpha =5\%$; (b) $\alpha =1\%$.}
\end{figure}

In summary for 5\% quantile forecasting: the VBDQ method is clearly and significantly superior for lower sample sizes of $T=250, 500$, and is
highly competitive for $T=1000, 2500$, regarding SAP. At these higher sample sizes, the BFDQ4 and DQ4 are marginally favoured, followed
by the VBDQ1, VBDQ4 and BFDQ1 tests. The out-performance in SAP for the VBDQ1 when $T \le 500$ is attributable to more
accurate detection of dependence in violations from the HS estimator, when the observed violation numbers were comparatively
small, compared to the DQ statistics, and simultaneously higher power than the BFDQ methods for comparatively higher violation numbers.
Further, the VBVQ method had significantly higher size-adjusted power than the BFVQ test and always had higher size-adjusted
power than the VQR test.

Table \ref{size1} shows the empirical estimates for size and then adjusted size across all the methods employed at $\alpha=0.01$ for
$T=250, 500$. Also shown are the empirical $5\%$ points for the methods. Again, empirical size for the BF methods is only reported as a
reference for comparison.

\begin{table}
\begin{center}
\caption{Size, empirical threshold and adjusted size for nominal 5\% tests of 1-step-ahead quantile forecasts at $\alpha=0.01$.
} \label{size1}
\begin{tabular}{lcccccccccccc}
\hline
 & \multicolumn{3}{c}{T=250} & \multicolumn{3}{c}{T=500}      \\
Method   &        Size  &          threshold &    adj. size &       Size  &           threshold &  adj. size    \\ \hline
DQ1      & \fbox{0.0600}&              9.164 &\cblue{0.0499}&      0.0727 &              12.202 & \fbox{0.0500} \\ [1.3pt]
BFDQ1    &       0.0039 & $3.411\times10^{5}$&\cblue{0.0499}&      0.0025 &             3766.82 & \fbox{0.0500} \\ [1.3pt]
VBDQ1    &       0.0000 & $3.184\times10^{5}$& \fbox{0.0500}&      0.0000 &  5.701$\times10^{5}$& \fbox{0.0500} \\ [1.3pt]
KRDQ1    &       0.0001 &             20.839 & \fbox{0.0500}&      0.0024 &              10.154 &\cblue{0.0499} \\ [1.3pt]
AQDQ1    &       0.0000 &             875.57 &\cblue{0.0499}&      0.0000 &             1053.38 & \fbox{0.0500} \\ [1.3pt]
DQ4      &       0.1128 &             28.657 &\cblue{0.0499}&      0.1691 &              21.411 & \fbox{0.0500} \\ [1.3pt]
BFDQ4    &       0.0110 & $2.633\times10^{8}$&\cblue{0.0499}&      0.0093 &  7.524$\times10^{7}$& \fbox{0.0500} \\ [1.3pt]
VBDQ4    &       0.0000 &$1.200\times10^{10}$&       0.0498 &      0.0000 & 2.578$\times10^{10}$& \fbox{0.0500} \\ [1.3pt]
VQR      &\cblue{0.0744}&              8.535 & \fbox{0.0500}&\fbox{0.0419}&               5.245 & \fbox{0.0500} \\ [1.3pt]
BFVQ     &       0.3783 &$1.725\times10^{-9}$&       0.0498 &      0.3397 &7.960$\times10^{-10}$&\cblue{0.0499} \\ [1.3pt]
VBVQ     &       0.3168 &$5.002\times10^{-9}$&\cblue{0.0499}&      0.2910 & 1.581$\times10^{-9}$& \fbox{0.0500} \\
\hline
\end{tabular}
\par\smallskip
\parbox{.9\textwidth}{ The correct, nominal thresholds are: VQR 5.99; DQ1 7.81; DQ4 12.59}
\end{center}
\end{table}

At $T=250$, the DQ1 test achieves closest to the true nominal size, with $6\%$, while the VQR and DQ4 are well over-sized.
Only the VQR and VBVQ1 tests achieve correct adjusted sizes of $5\%$; though the others are only very marginally under-sized.
For $T=500$, the DQ1 and DQ4 are both over-sized, the VQR is under-sized and no method is very close to the nominal level.
Table \ref{size11} shows the empirical estimates for size and then adjusted size across all the methods employed at
$\alpha=0.01$ for $T=1000,2500$. Also shown are the empirical $5\%$ points for the methods across the 25000
replications used to calculate the adjusted size and size-adjusted power.

\begin{table}
\begin{center}
\caption{Size, empirical threshold and adjusted size for nominal 5\% tests of 1-step-ahead quantile forecasts at $\alpha=0.01$.
} \label{size11}
\begin{tabular}{lcccccc}
\hline
 & \multicolumn{3}{c}{T=1000} & \multicolumn{3}{c}{T=2500}                    \\
Method   &        Size  &          threshold  &    adj. size &        Size  &           threshold &   adj. size    \\ \hline
DQ1      &\cblue{0.0802}&              9.505  & \fbox{0.0500}&\cblue{0.0582}&               8.534 & \fbox{0.0500}   \\ [1.3pt]
BFDQ1    &       0.0032 & 3.073$\times10^{5}$ & \fbox{0.0500}&       0.0003 &  2.954$\times10^{6}$&\cblue{0.0499}   \\ [1.3pt]
VBDQ1    &       0.0000 & 1.808$\times10^{6}$ & \fbox{0.0500}&       0.0000 &  7.679$\times10^{6}$& \fbox{0.0500}   \\ [1.3pt]
KRDQ1    &       0.0003 &             365.249 & \fbox{0.0500}&       0.0000 &             7876.47 &\cblue{0.0499}   \\ [1.3pt]
AQDQ1    &       0.00004&             3384.83 & \fbox{0.0500}&       0.0000 &            11461.53 & \fbox{0.0500}   \\ [1.3pt]
DQ4      &       0.1045 &             16.645  & \fbox{0.0500}&       0.1038 &              15.495 & \fbox{0.0500}   \\ [1.3pt]
BFDQ4    &       0.0022 & 1.263$\times10^{9}$ & \fbox{0.0500}&       0.0002 & 3.708$\times10^{11}$& \fbox{0.0500}   \\ [1.3pt]
VBDQ4    &       0.0000 & 4.319$\times10^{11}$& \fbox{0.0500}&       0.0000 & 5.647$\times10^{12}$& \fbox{0.0500}   \\ [1.3pt]
VQR      & \fbox{0.0406}&              5.177  & \fbox{0.0500}& \fbox{0.0473}&               5.772 & \fbox{0.0500}   \\ [1.3pt]
BFVQ     &       0.2971 &8.438$\times10^{-10}$& \fbox{0.0500}&       0.2390 & 5.036$\times10^{-8}$& \fbox{0.0500}   \\ [1.3pt]
VBVQ     &       0.2466 &2.001$\times10^{-9}$ & \fbox{0.0500}&       0.1667 & 9.025$\times10^{-8}$& \fbox{0.0500}   \\ [1.3pt]
\hline
\end{tabular}
\par\smallskip
\parbox{.9\textwidth}{ The correct, nominal 5\% thresholds are: VQ 5.99; LR1, DQ1 7.81; DQ4 12.59}
\end{center}
\end{table}

At $T=1000$, again no method is close to the nominal 5\% level: DQ1, DQ4 are over and VQR is under-sized; all methods
have exactly nominal adjusted size. For $T=2500$ the VQR test has the closest to nominal size ($4.9\%$), whilst
all methods achieve an adjusted size exactly equal to nominal.

\begin{figure}[thp]
     \centering
      \includegraphics[width=1.0\textwidth]{pdq1000_final.png}
\caption{\label{pdq1000} Rejection rates for the VBDQ1 (dark grey diamond), BFDQ1 (medium grey circle), BFDQ4 (light grey circle),
DQ1 (black square) and DQ4 (grey square) statistics against number of violations under the HS quantile estimator at $T=1000$;
(a) 5\% HS; (b) 1\% HS.}
\end{figure}

\begin{figure}%[htp]
     \centering
      \includegraphics[width=0.8\textwidth]{pvq1000.png}
\caption{\label{pvq1000} Rates of rejection for the VBVQ (dark grey diamond), BFVQ (light grey circle) and VQR (black square)
statistics against the number of violations under the HS quantile estimator at $T=1000$; (a) 5\% HS; (b) 1\% HS.}
\end{figure}

Table \ref{power1} shows the empirical estimates for power and size-adjusted power, as well as the estimated standard errors for
size-adjusted power, across all the methods at $\alpha=0.01$. At $T=250$ two methods stand out regarding size-adjusted power: VBDQ1 and BFDQ1;
VBVQ is next best with $\approx 39\%$ and DQ1 is next with $\approx 34\%$. To examine power in more detail, consider
Figure \ref{powerDQBF250}(b), showing minus the logarithm of the VBDQ1, BFDQ1 statistics and the logarithm of the DQ1 statistic, 
plot against the number of violations under the alternative hypothesis. As is logical, with an expected number of only 2.5 
violations under the null, all methods always reject the HS estimator only for very high numbers of violations (e.g. 8 and above for DQ1, 
VBDQ1, 12 and above for BFDQ1).
Gerlach et al (2014) found the comparative strong performance of BFDQ1 is due to a much higher rate of model rejection for
low violation numbers, here 1-4, as shown in Figure \ref{pdqBF250}(b) (comparing VBDQ1, BFDQ1 and DQ1 in terms of rejection rates
against number of violations), all of which occur highly frequently at $T=250$ and 1\% HS forecasting. The DQ1 has higher power than
BFDQ1 at 5-11 violations, but these violation numbers are far less likely to occur at $T=250$. The VBDQ1 test mimics the 
higher rejection rates of BFDQ1 for 1-4 violations, and also mimics the higher rejection rates of DQ1 for 5-11 violations, thus ensuring 
it is significantly higher in SAP than both DQ1 and BFDQ1 in this case. We run all 55 pairwise
likelihood ratio tests for equality of SAP: 53 of these tests had p-values $\approx 0$, suggesting clear
statistically significantly different SAPs, with the differences between VBDQ4 and VBVQ (p-val $= 0.15$) and between BFDQ4 and BFVQ
(p-val $= 0.26$) in SAPs being insignificant from 0.


%\begin{footnotesize}
\begin{table}
\begin{center}
\caption{Power and size-adjusted power for nominal 5\% tests of 1-step-ahead quantile forecasts at $\alpha=0.01$ and $T=250, 500$.
}\label{power11}
\begin{tabular}{lcccccc}
\hline
 & \multicolumn{3}{c}{T=250} & \multicolumn{3}{c}{T=500}                   \\
Method   &     Power   &  Size-adj.  &   SE   &       Power &  Size-adj.  &  SE  \\ \hline
DQ1      &\cblue{0.384}&       0.344 & 0.0030 &\cblue{0.468}&       0.357 & 0.0030 \\ [1.3pt]
BFDQ1    &       0.042 &\cblue{0.475}& 0.0032 &       0.034 &       0.131 & 0.0021 \\ [1.3pt]
VBDQ1    &       0.001 & \fbox{0.509}& 0.0032 &       0.0001& \fbox{0.447}& 0.0031 \\ [1.3pt]
KRDQ1    &       0.013 &       0.146 & 0.0022 &       0.025 &       0.123 & 0.0021 \\ [1.3pt]
AQDQ1    &       0.003 &       0.188 & 0.0025 &       0.004 &       0.216 & 0.0026 \\ [1.3pt]
DQ4      &\cblue{0.443}&       0.293 & 0.0029 &\cblue{0.629}&\cblue{0.420}& 0.0031 \\ [1.3pt]
BFDQ4    &       0.050 &       0.279 & 0.0028 &       0.047 &       0.256 & 0.0028 \\ [1.3pt]
VBDQ4    &       0.000 &\cblue{0.372}& 0.0031 &       0.000 &       0.340 & 0.0030 \\ [1.3pt]
VQR      &       0.104 &       0.074 & 0.0017 &       0.063 &       0.071 & 0.0016 \\ [1.3pt]
BFVQ     &\cblue{0.688}&       0.255 & 0.0028 &\cblue{0.804}&       0.343 & 0.0030 \\ [1.3pt]
VBVQ     &\cblue{0.751}&\cblue{0.385}& 0.0031 & \fbox{0.805}&\cblue{0.420}& 0.0031 \\ [1.3pt]
\hline
\end{tabular}
\par\smallskip
\parbox{.9\textwidth}{ SE is $\sqrt{\frac{p(1-p)}{25000}}$, where $p$ is size-adjusted power.}
\end{center}
\end{table}
%\end{footnotesize}

\begin{table}
\begin{center}
\caption{Power and size-adjusted power for nominal 5\% tests of 1-step-ahead quantile forecasts at $\alpha=0.01$ and $T=1000, 2500$.
}\label{power12}
\begin{tabular}{lcccccc}
\hline
  \multicolumn{3}{c}{T=1000} & \multicolumn{3}{c}{T=2500}                   \\
Method   &     Power   &  Size-adj.  &   SE   &       Power &  Size-adj.  &    SE    \\ \hline
DQ1      &\cblue{0.666}&\cblue{0.577}& 0.0031 &\cblue{0.933}&\cblue{0.913}& 0.0018 \\ [1.3pt]
BFDQ1    &       0.029 &       0.370 & 0.0031 &       0.056 &       0.832 & 0.0024 \\ [1.3pt]
VBDQ1    &       0.001 &\cblue{0.615}& 0.0031 &       0.010 &\cblue{0.935}& 0.0016 \\ [1.3pt]
KRDQ1    &       0.015 &       0.276 & 0.0028 &       0.039 &       0.675 & 0.0030 \\ [1.3pt]
AQDQ1    &       0.006 &       0.412 & 0.0031 &       0.041 &       0.759 & 0.0027 \\ [1.3pt]
DQ4      &\cblue{0.757}&\cblue{0.618}& 0.0031 &\cblue{0.950}&\cblue{0.913}& 0.0018 \\ [1.3pt]
BFDQ4    &       0.036 &       0.356 & 0.0030 &       0.050 &       0.821 & 0.0024 \\ [1.3pt]
VBDQ4    &       0.000 & \fbox{0.637}& 0.0030 &       0.0002&\cblue{0.943}& 0.0015 \\ [1.3pt]
VQR      &       0.151 &       0.170 & 0.0024 &       0.562 &       0.572 & 0.0031 \\ [1.3pt]
BFVQ     & \fbox{0.909}&\cblue{0.559}& 0.0031 & \fbox{0.992}& \fbox{0.953}& 0.0013 \\ [1.3pt]
VBVQ     &\cblue{0.880}&\cblue{0.587}& 0.0031 &\cblue{0.983}&\cblue{0.928}& 0.0016 \\ [1.3pt]
\hline
\end{tabular}
\par\smallskip
\parbox{.9\textwidth}{ SE is $\sqrt{\frac{p(1-p)}{25000}}$, where $p$ is size-adjusted power.}
\end{center}
\end{table}

Gerlach et al (2014) found that the performance of the BFDQ1, BFDQ4 methods compared unfavourably with that of
DQ1, DQ4 for $T>250$; our results concur with that. However, the VBDQ1 statistic has no such poor comparison.
For $T=500$, the method with highest SAP is the VBDQ1 ($45\%$), followed by the
DQ4 and VBVQ ($42\%$ SAP). In this case, the BFDQ1 and BFDQ4 statistics out-performed the DQ1 and DQ4 respectively
only when 1-4 violations were observed, but these are not very likely when $T=500$ at $\alpha=1\%$ forecasting, and the
out-performance of the DQ statistics for the more frequently occurring violation numbers of 5-12 was enough to
overall clearly out-perform both BFDQ statistics here. Again the VBDQ1 statistic mimicked the higher rejection
rates of the DQ1 and DQ4 statistics for higher numbers of violations, and the higher rejection rates of the BFDQ1 for smaller
numbers of violations, ensuring it was the best method re SAP overall. The pairwise tests for equality of size-adjusted
power yielded 53 p-values $<0.05$ and $\approx 0$, with 61 of these $\approx 0$ (highest of these $0.0046$), with only the two
differences between DQ4 and VBVQ and between VBDQ4 and VBVQ giving insignificant p-values of $0.82, 0.33$ respectively (observed differences
$0.0008, 0.003$ respectively).

At $T=1000$ the VBDQ4 test has the highest SAP, closely followed by the VBDQ1 and DQ4 statistics. The pairwise tests for equality
of SAP yield 54 p-values $\approx 0$ (highest 0.0025), with only the difference between VBDQ1 and DQ4 being
insignificant, with p-value $0.273$. The test of equality between the top-ranked VBDQ4 and DQ4 SAPs has p-value $0$. Figure \ref{pvq1000}
plot (b) shows rejection rates for the VBVQ, BFVQ and VQR methods, for $1\%$ VaR forecasting. Clearly, at almost all violation numbers 
both the VBVQ and BFVQ have comparable and much higher rejection rates compared to the VQR test, with the VBVQ having marginally 
higher SAP than the BFVQ at most violation numbers.

At $T=2500$ the BFVQ test recorded the highest SAP with 95\%, closely followed by the VBDQ4 method with 94\%, then
the VBDQ1 and VBVQ, with $\sim 92-93\%$. The pairwise tests for equality of size-adjusted
power yielded 54 p-values $<0.05, \approx 0$ (highest of these $0.00006$), and only the difference between
DQ1 and DQ4 is insignificant, with p-value $0.85$ (observed difference $0.00006$).
The test of equality between the top-ranked BFVQ and VBDQ4 SAPs has a p-value
of $2.62 \times 10^{-7}$. At all sample sizes except $T=2500$, the VBVQ method had higher
size-adjusted power than the BFVQ method, and it always had much higher size-adjusted power than the VQR test.

Overall, for 1\% quantile forecasting, the results clearly favour the VBDQ1 method for $T \le 500$. At $T=1000$ the VBDQ4 test is marginally
favoured over the DQ4 and VBDQ1 methods, which still perform strongly. Finally, at $T=2500$, the BFVQ method is favoured regarding
size-adjusted power, closely followed by the VBDQ4, then the VBVQ1, DQ1 and DQ4 methods.

When comparing the VB or BF version of each test with its frequentist counterpart (e.g. VBDQ vs BFDQ1 vs DQ1, etc),
at the 5\% quantile level the results are very clear: at each sample size $T$ the Bayesian versions of the test had
higher size-adjusted power, often only marginally but sometimes much, much higher, than its frequentist competitor, for all the
tests considered (except VQR, BFVQ at $T=1000$). This is a very strong and clear result in favour of the Bayesian methods. Further,
the VBVQ method always out-performed the BFVQ, whilst the VBDQ method out-performed the BFDQ1 method for $T \le 500$.

For 1\% quantile forecasting, the comparison is not so clear: while the results clearly favour the VBDQ method for
$T \le 500$, at $T=1000$ the DQ4 test is marginally favoured over the VBDQ method. Finally, at $T=2500$, the
BFVQ method is favoured regarding size-adjusted power, followed closely by both the DQ and VB methods. However, the
VBVQ method always had higher size-adjusted power than the VQR test. Further, the VBVQ method outperformed the BFVQ for $T \le 1000$,
while the VBDQ always had higher size-adjusted power than its BFDQ1 counterpart.

\subsection{Accuracy of VB approximations}
Marginal likelihood estimates are typically hard to assess for accuracy, since the quantity is generally intractable analytically.

For the BF statistics based on the VQ regression, the integral is only in two dimensions, so fairly simple methods can be employed to get reliable,
accurate estimates of the marginal likelihood. The integral ${ \int \int p({\bf u}|\utwi{\beta}) p(\utwi{\beta}) d \beta_0 d \beta_1 }$ is simply
$E [ p({\bf u}|\utwi{\beta}) ]$ over the prior distribution. A simulation estimate is then
$$
{ \int \int p({\bf u}|\utwi{\beta}) p(\utwi{\beta}) d \beta_0 d \beta_1 } \approx \frac{1}{J} \sum_{j=1}{J} p({\bf u}|\utwi{\beta}^{[j]}) \,\, ,
$$
where $\utwi{\beta}^{[j]}$ is simulated directly from $p(\utwi{\beta}) \sim N(0,cI)$. An acceptable level of MC error in this estimate
is achieved when $J=200000$; this estimator is denoted simVQ. A second estimate, denoted MetVQ, employs the Metropolis method to
simulate an MCMC sample from the posterior; Gelfand and Dey (1990) illustrated that:
$$
$$
estimates the marginal likelihood required. The estimator may have infinite variance and be unstable due to the inverse densities
involved; however, in the case the estimator seems to have stable, good properties and thus is used as a second comparison method.

Table xxx illustrates some statistics regarding the VBVQ, BFVQ, MetVQ and simVQ statistics, estimated over xx samples of data simulated
from the model:
\begin{eqnarray*}
\sigma_t^2 = 0.1 + 0.1y_{t-1}^2 + 0.85\sigma_{t-1}^2 \,\,;\,\, y_t = \sigma_t \epsilon_t \,;\, \epsilon_t \sim N(0,1) \\
\end{eqnarray*}
where $\mbox{VaR}_{t,\alpha} = \sigma_t \Phi^{-1}(\alpha)$. Table yyy shows the size-adjusted power of these methods.

Figure xxx  shows ...  Clearly, the entire distribution of the BF statistics is shifted by the VB estimator giving a lower bound.
However, the VB method seems to give a more stable, reliable estimate, leading to a more (size-adjusted) powerful test, that can be calculated
in a smaller time. The AQ and KR approximations give more accurate estimates of the marginal likelihood on average, but their inherent
approximation error leads to lower power BF tests.



For the BF statistics based on probit regression, VBDQ1, AQDQ1, KRDQ1, the associated accuracy of the marginal likelihood estimates is
particularly challenging to assess. The gold standard in this model for marginal likelihood estimation is the method of Chib (1995), which
exploits the fact that the marginal likelihood is the ratio of the posterior to the likelihood, which can be evaluated at any value of
the parameters. This method relies on an MCMC sample from the probit regression model, to estimate the posterior mean of $\beta$, integrate out
the latent series $y*$, and .... The MCMC sample needs many iterates, especially when the number of violations is a small proportion
of $T$: always the case when $\alpha=0.01$. The method is not exact and has simulation or Monte Carlo (MC) error. To facilitate its accuracy,
we start the MCMC sample at the posterior mode, however 20000 iterates are still needed to achieve an acceptable level of MC error.

Table xxx illustrates some statistics regarding the VBDQ1, AQDQ1, KRDQ1, ChibDQ1 statistics, estimated over the same xx samples of data as above.
Table yyy shows the size-adjusted power of these methods.

Figure xxx  shows ...



%\section{Discussion}
%When detecting the incorrect HS estimator of 1\% and 5\% quantiles using a range of competing tests/methods, fairly similar stories can be told at
%each quantile level. First, the VBDQ method is almost always prevalent at or near the top of the rankings regarding size-adjusted power.
%On the contrary, the VQR method always performed towards the bottom on this aspect. The reported performance of the VQR test
%in terms of size-adjusted power is slightly worse than the results in Gaglianone et al. (2011), though
%nearly comparable, but agree with those in Gerlach et all (2014).

%The relatively poor performance of the VQR, compared to the BFVQ, test bears more examination. Figure \ref{VQ1000} shows two times the
%logarithm of the BFVQ (upper) and VQ (lower) test statistics, plot against the number of violations, under the null hypothesis (black circles)
%and also under the HS quantile estimator (grey diamonds), at $n=1000$. Also shown are two times the logarithm of the
%empirical 5\% points of the BFVQ (upper) and VQR (lower) statistics; points above these lines represent rejections of the null hypothesis.
%It is immediately apparent that slightly more violations tend to occur under the HS estimator than under the null, the latter having a distribution
%shifted to the right compared to the former (the ratio of means is 1.34 as mentioned previously). Further, it is clear that the BFVQ
%statistic has a clearly distinguished distribution of values, typically higher, under the HS estimator (grey diamonds) compared to that
%under the null; this leads to the observed 55.7\% size-adjusted power of the BFVQ method. On the contrary, the VQR test statistic does not
%have a clearly distinguished distribution of values under the HS estimator compared to that the null, leading to its very low size adjusted power
%(17.4\%). Similar plots, not shown, occur at $n=250, 500, 2500$.
%
%\begin{figure}%[htp]
%     \centering
%      \includegraphics[width=0.95\textwidth]{VQ1000.png}
%\caption{\label{VQ1000} Two times the logarithm of the (a) BFVQ and (b) VQ test statistics, plot against the number of violations,
%under the null hypothesis (black circles) and the HS quantile estimator (grey triangles) at $n=1000$.
%The horizontal lines are two times the logarithm of the empirical 5\% points of the BFVQ (a) and VQR (b) statistics.}
%\end{figure}



\section{Empirical study}
We briefly report the results of a large empirical study here. Seven daily financial time series: prices, exchange rates or
financial indices, are considered, in each case converting these to daily percentage log returns. The seven series are: the US
S\&P500 index, the AORD, FTSE100 and Hang Seng indices, the AU US exchange rates, the EU US exchange rates and IBM asset prices.
The initial sample period is specifically from January 2, 1998 to December 15, 2005, approximately 2000 days in each case. The forecast
period is December 16, 2005 to January 15, 2010, covering close to 1000 trading days in each market, and including the well-known
global financial crisis (GFC) period.

One-step-ahead forecasts of VaR at 5\% and 1\% quantile levels are estimated under a range of competing models and methods, for each day in
the forecast period. Forecasts for each of four types of heteroskedastic model: the GARCH of Bollerslev (1986), the GJR-GARCH of
Glosten et al (1993) the Threshold (T-)GARCH in Chen, Gerlach and So (2006) and a smooth transition (ST-)GARCH as in Chen and Gerlach (2008)
are estimated employing the MCMC methods of Chen et al. (2012). Each of these specifications is estimated under five types of error
distribution: Gaussian, Student-t, the skewed Student-t of Hansen (1994), the Asymmetric Laplace (AL) of Chen et al (2012) and the
Two-sided Weibull (TW) of Chen and Gerlach (2013). This gives 20 models generating VaR forecasts at 5\% and 1\% quantile levels for 1000 days.
Each day's forecast used the last $\sim 2000$ days as the estimation period for each model, thus a rolling window approach is used.
Also considered are the non-parametric 50 day and 250 day sample percentile HS methods, thus giving a total of 22 models or methods.
Estimation results are not shown to save space, since only the test results are directly relevant to this paper; it is expected that most models
and methods will be rejected since the data includes the GFC period, where that outcome is common;
but models and methods that can better captured highly changing volatility and fat-tailed returns will be rejected the least,
across the seven series.

Tables \ref{reject5} and \ref{reject1} show the number of series, out of 7, that each model or method of VaR estimation was rejected in,
using the DQ1, DQ4, BFDQ1, BFDQ4, VBDQ, VQR, BFVQ and VBVQ tests for 5\% and 1\% VaR forecasting, respectively.
The tests were conducted at the 5\% level using the empirical cut-offs in Tables \ref{size5}-\ref{size11} above.

\begin{table}
\begin{center}
\caption{Number of rejections for each model across 7 series for 1\% VaR forecasting.}\label{reject5}
\begin{tabular}{lcccccccc}
\hline
    Method &    DQ1 &    DQ4 &  BFDQ1 &  BFDQ4 &   VBDQ &   VQ   &   BFVQ &   VBVQ   \\ \hline
       G-n &      6 &      5 &\fbox{0}&      1 &      3 &      2 &      4 &      4   \\
     GJR-n &      4 &      5 &\fbox{0}&\fbox{0}&      4 &      4 &      5 &      6   \\
      TG-n &      6 &      5 &      1 &\fbox{0}&      4 &      3 &      5 &      5   \\
      ST-n &      6 &      6 &      1 &\fbox{0}&      5 &      4 &      6 &      6   \\
       G-t &      1 &      3 &      1 &      3 &\fbox{0}&      2 &      1 &      1   \\
     GJR-t &\fbox{0}&\fbox{0}&\fbox{0}&      2 &      1 &      1 &      1 &      2   \\
     TG-t  &\fbox{0}&\fbox{0}&      1 &      2 &      1 &      1 &      2 &      2   \\
      ST-t &      2 &      1 &      1 &      2 &\fbox{0}&      2 &      2 &      2   \\
     G-SKT &      1 &      1 &\fbox{0}&\fbox{0}&\fbox{0}&\fbox{0}&      1 &      1   \\
   GJR-SKT &      1 &      1 &\fbox{0}&\fbox{0}&\fbox{0}&      1 &      1 &      1  \\
    TG-SKT &      1 &      1 &\fbox{0}&      1 &\fbox{0}&\fbox{0}&      2 &      2   \\
    ST-SKT &      1 &      1 &\fbox{0}&\fbox{0}&\fbox{0}&      1 &      2 &      2   \\
      G-AL &      1 &      2 &      2 &      4 &      1 &      3 &      3 &      3   \\
    GJR-AL &      1 &      2 &      2 &      1 &      1 &      5 &\cred{7}&      6   \\
     TG-AL &      1 &      2 &      3 &      2 &      1 &      5 &      5 &      4   \\
     ST-AL &      1 &      2 &      2 &      2 &      1 &      2 &      4 &      4   \\
      G-TW &      1 &\fbox{0}&\fbox{0}&      1 &\fbox{0}&      1 &      1 &      1   \\
    GJR-TW &\fbox{0}&\fbox{0}&      1 &      1 &      2 &      1 &      2 &      2   \\
     TG-TW &      1 &\fbox{0}&\fbox{0}&      2 &\fbox{0}&      1 &      1 &      1   \\
     ST-TW &      1 &\fbox{0}&\fbox{0}&      2 &\fbox{0}&\fbox{0}&      1 &      1   \\
     HS250 &\cred{7}&\cred{7}&      3 &      6 &\cred{7}&      5 &\cred{7}&\cred{7}   \\
      HS50 &\cred{7}&\cred{7}&\cred{7}&      3 &\cred{7}&\cred{7}&\cred{7}&\cred{7}  \\ \hline
\end{tabular}
\end{center}
\end{table}

For 1\% VaR forecasting, the various tests often have quite different numbers of rejections for each model. The DQ tests reject all
models with Gaussian errors in most series, as do the VBDQ, BFVQ and VBVQ tests; however, the BFDQ tests seem to have lower power
at rejecting these, perhaps reflecting their comparatively low power in the simulation study. A similar result occurred for
the HS methods, where the VBDQ and VBVQ also rejected these in all 7 series. Broadly in agreement with Chen et al (2012),
the HS and models with Gaussian errors can be rejected as adequate 1\% VaR forecasters, whilst models with TW and SKT errors are
rejected the least across the various tests, and thus seem most suitable and accurate as 1\% VaR forecasters for these series.


For 5\% VaR forecasting, the results are qualitatively similar. One difference is that models with Gaussian errors are rejected
less frequently for 5\% VaR forecasting; whilst models with AL errors are added to those that are rejected only in a few of the series,
across the various tests.

\begin{table}
\begin{center}
\caption{Number of rejections for each model across 7 series for 5\% VaR forecasting.}\label{reject1}
\begin{tabular}{lccccccccc}
\hline
    Method &    DQ1 &    DQ4 &  BFDQ1 &  BFDQ4 &   VBDQ &   VQ   &   BFVQ &   VBVQ   \\ \hline
       G-n &\fbox{0}&      3 &\fbox{0}&      1 &\fbox{0}&\fbox{0}&\fbox{0}&\fbox{0}  \\
     GJR-n &      2 &      3 &\fbox{0}&      1 &      1 &      1 &      2 &      2   \\
      TG-n &      3 &      3 &      1 &      1 &      1 &      2 &      2 &      2   \\
      ST-n &      3 &      3 &      1 &      1 &      1 &      1 &      2 &      3   \\
       G-t &      1 &      3 &      1 &      3 &      1 &      2 &      2 &      1   \\
     GJR-t &      1 &      2 &      1 &      2 &      1 &      1 &      1 &      1   \\
      TG-t &      1 &      2 &      1 &      2 &      1 &      2 &      1 &      1   \\
      ST-t &      1 &      2 &      1 &      2 &      1 &      3 &      1 &      1   \\
     G-SKT &\fbox{0}&      3 &\fbox{0}&      1 &\fbox{0}&\fbox{0}&\fbox{0}&\fbox{0}  \\
   GJR-SKT &      1 &      3 &\fbox{0}&\fbox{0}&\fbox{0}&      1 &\fbox{0}&\fbox{0}  \\
    TG-SKT &      2 &      3 &\fbox{0}&\fbox{0}&\fbox{0}&      2 &      1 &      3     \\
    ST-SKT &      1 &      3 &\fbox{0}&\fbox{0}&\fbox{0}&      1 &      1 &      2     \\
      G-AL &\fbox{0}&      2 &\fbox{0}&      2 &\fbox{0}&      1 &\fbox{0}&\fbox{0}   \\
    GJR-AL &      1 &      2 &      1 &      1 &\fbox{0}&      1 &      1 &\fbox{0}   \\
     TG-AL &\fbox{0}&      2 &      1 &      1 &\fbox{0}&      1 &      1 &      1     \\
     ST-AL &\fbox{0}&      2 &\fbox{0}&      1 &\fbox{0}&      1 &\fbox{0}&      1     \\
      G-TW &\fbox{0}&      2 &\fbox{0}&      1 &\fbox{0}&\fbox{0}&\fbox{0}&\fbox{0}     \\
    GJR-TW &\fbox{0}&      2 &\fbox{0}&      1 &\fbox{0}&\fbox{0}&\fbox{0}&      1     \\
     TG-TW &\fbox{0}&      2 &\fbox{0}&\fbox{0}&\fbox{0}&\fbox{0}&\fbox{0}&      1     \\
     ST-TW &      1 &      2 &\fbox{0}&\fbox{0}&\fbox{0}&      1 &\fbox{0}&\fbox{0}     \\
     HS250 &\cred{7}&      6 &\cred{7}&\cred{7}&\cred{7}&\cred{7}&\cred{7}&\cred{7}     \\
     HS50  &      6 &      5 &\cred{7}&\cred{7}&      6 &\cred{7}&\cred{7}&\cred{7}   \\  \hline
\end{tabular}
\end{center}
\end{table}



\section{Conclusion}
Two variational Bayes methods are developed for assessing and testing forecast accuracy for dynamic quantile forecasts. The
tests are analogues of the VQR and DQ tests commonly applied in this area. The proposed methods do not depend on the
estimation technique, nor on the model or method used, to generate the quantile forecasts.
In a simulation study, at both $\alpha=0.01, 0.05$ quantile levels, these methods performed favourably, regarding
size-adjusted power, compared to their competing Bayesian and frequentist analogues.

\vspace{3cm}


\noindent{\bf Acknowledgement}
The authors thank Declan Walpole for coding up the VQR test.

\newpage
\noindent{\bf References}

\begin{description}
\item
Berkowitz, J., Christoffersen, P., and Pelletier, D. (2011), ``Evaluating Value-at-Risk Models with Desk-Level Data,''
{\em Management Science}, {\bf 57}, 2213-2227.

\item Bollerslev, T. (1986). ``Generalized Autoregressive Conditional Heteroskedasticity,'' {\em Journal of Econometrics},
{\bf 31}, 307-327.

\item Brown, L. D., Cai, T. T., and DasGupta, A. (2001) ``Interval Estimation for a Binomial Proportion,''
{\em Statistical Science}, {\bf 16}, 101-117.

\item Chen, C. W. S., Gerlach, R., Hwang, R. B. K., and McAleer, M. (2012), ``Forecasting Value-at-Risk
using nonlinear regression quantiles and the intra-day range,'' {\em International Journal of Forecasting}, {\bf 28}, 557-574.

\item
Chen, C. W. S., Gerlach, R., Lin, E. M. H., and Lee, W.C.W. (2012), ``Bayesian forecasting for financial risk management, pre and post the
global financial crisis,'' {\em Journal of Forecasting}, {\bf 31}, 661-687.

\item
Chen, C. W. S., and  So, M. K. P. (2006). ``On a threshold heteroscedastic model,'' {\em International Journal of Forecasting}, {\bf 22},
73-89.

\item
Chen, Q., and Gerlach, R. (2013). ``The two-sided Weibull distribution and forecasting financial tail risk,''
{\em International Journal of Forecasting}, {\bf 29}, 527-540.

\item
Chen, Q., Gerlach, R. and Lu, Z. (2012), ``Bayesian Value-at-Risk and expected shortfall forecasting via the asymmetric
Laplace distribution,'' {\em Computational Statistics \& Data Analysis}, 1st Issue of Annals of Computational and Financial
Econometrics, {\bf 56}, 3498-3516.

\item Christoffersen, P. F. (1998), ``Evaluating interval forecasts,'' {\em International Economic Review}, {\bf39}, 841-862.

\item Engle R. F., and Manganelli S. (2004), ``CAViaR: Conditional autoregressive value at risk by regression quantiles,''
{\em Journal of Business and Economic Statistic}, {\bf 22}, 367- 381.

\item Gaglianone, W. P., Lima, L. R., Linton, O., and Smith, D. R. (2011), ``Evaluating Value-at-Risk models via quantile regression,''
    {\em Journal of Business \& Economic Statistics}, {\bf 29}, 150-160.

\item
Gelman, A., Carlin, J. B., Stern, H. S., and Rubin, D. B. (2005), Bayesian Data Analysis (2nd ed.), Boca Raton, FL: Chapman \& Hall.

\item Gerlach, R., and Chen, C. W. S. (2008), ``Bayesian inference and model comparison for asymmetric smooth transition
   heteroskedastic models,'' {\em Statistics and Computing}, {\bf 18}, 391-408.

\item  Gerlach, R., Chen, C.W.S and Lin EMH (2014), ``Bayesian Assessment of Dynamic Quantile Forecasts'', The University of
    Sydney Business Analytics Working Paper Series, http://sydney.edu.au/business/business_analytics/research/working_papers.

\item
Girolami, M. and Rogers, S. (2006) ``Variational Bayesian multinomial probit regression with Gaussian process priors,''
{\em Neural Computation} 18, {\bf 8}, 1790-1817.

\item Glosten, L. R., Jagannathan, R., and Runkle, D. E. (1993). ``On the relation between the expected value and the volatility
of the nominal excess return on stocks,'' {\em Journal of Finance}, {\bf 487}, 1779-1801.

\item Hansen, B., (1994). ``Autoregressive conditional density estimation,'' {\em International Economic Review}. {\bf 35}, 705–730.

\item Hoogerheide, L. F., and van Dijk, H.K. (2010), ``Bayesian forecasting of Value at Risk and expected shortfall using
adaptive importance sampling", {\em International Journal of Forecasting}, {\bf 26}, 231-247.

\item
Koenker, R., and Machado, J. A. F. (1999) ``Goodness of fit and related inference for quantile regression,'' {\em Journal of the American
Statistical Association}, {\bf 94}, 1296-1310.

\item
Kupiec, P. (1995) ``Techniques for verifying the accuracy of risk measurement models,'' {\em Journal of Derivatives},
{\bf 2}, 173-84.

\item
Ormerod, J. and Wand, M. (2010) ``Explaining variational approximations,'' {\em American Statistician}, 64, {\bf 2}, 140-153.

\item
Smith, M., and Kohn, R. (1996). ``Non-parametric Regression Using Bayesian Variable Selection,'' {\em Journal of Econometrics},
{\bf 75}, 317-343.

\item
Tzikas, D., Likas, A.  and Galatsanos, N. ``Variational Bayesian sparse kernel-based blind image deconvolution with student's-t priors,'' IEEE Transactions on Image Processing, vol. 18, no. 4, pp. 753-764.

\item Tuyl, F., and Gerlach, R., and Mengersen, K. (2008). ``Inference for Proportions in a 2 x 2 Contingency Table: HPD or not HPD?,''
{\em Biometrics}, {\bf 64}, 1293-1296.



\end{description}

\input{appendix_030715}

\end{document}
